\documentclass[addpoints,12pt]{exam}
\usepackage[spanish]{babel}
%\usepackage[latin1]{inputenc}
\usepackage[utf8x]{inputenc}
\pagestyle{empty}
\begin{document}
\begin{center}
\fbox{\fbox{\parbox{5.5in}{\centering {\LARGE EXAMEN TIPO A}\\El ejercicio se realizará en eclipse. Usa un workspace distinto al que usas de forma habitual. Crea un proyecto de java nuevo denominado Examen y posteriormente un package denominado ejercicio, donde desarrollar los programas}}}
\end{center}
\vspace{0.1in}
%\makebox[\textwidth]{Nombre:\enspace\hrulefill}
\begin{questions}
\question Queremos hacer un programa para registrar los datos de distintas ciudades para esto debemos implementar las siguientes clases:
\begin{description}
\item[clase Ciudad] Qué tiene los siguientes atributos y métodos:
\begin{itemize}
\item Nombre de la ciudad
\item Latitud
\item Longitud.
\item Nº de habitantes.
\item Un constructor que inicialice dichos atributos.
\item Los correspondientes \emph{getters}
\item Sobreescribe el método \emph{toString()} para que cuando tratemos la referencia de estos objetos como \emph{String}, nos muestre los datos con el siguiente formato:\\
Ciudad: (nombre de la ciudad) - Nº habitantes: (nº habitantes)
\end{itemize}
\item[clase Pais] Qué tiene los siguientes atributos y métodos:
\begin{itemize}
\item Nombre del pais.
\item Coleccion dinamica de ciudades.
\item Un constructor que inicialice atributo nombre.
\item Un método que nos sirva para añadir objetos \emph{Ciudad} a la coleccion de ciudades.
\item Un método que nos devuelva el objeto \emph{Ciudad} con mas habitantes.
\item Un método que nos diga si una ciudad existe o no. Pasamos como argumento el nombre de la ciudad, es valido pasar el nombre tanto en mayusculas como en minusculas.
\item Un método que nos devuelva la población media de los distintos objetos \emph{Ciudad} que posee la lista de dichos objetos.
\item Sobreescribe el método \emph{toString} a tu gusto.
\item Puedes añadir cualquier método que veas necesario, aunque no es necesario para el desarrollo del ejercicio. Explica el porqué de su uso.
\end{itemize}
\item[LatitudException] Nos sirve para el caso que intentemos crear un objeto \emph{Ciudad} con un valor de la latitud errónea. La latitud es la distancia angular entre la línea ecuatorial (el ecuador), y un punto determinado de la Tierra. Sus valores pueden oscilar entre -90 (valor del polo sur) a 90 (valor del polo norte) pasando por 0 (valor del ecuador).
\item[LongitudException] Nos sirve para el caso que intentemos crear un objeto \emph{Ciudad} con un valor de la longitud errónea. La longitud expresa la distancia angular entre un punto dado de la superficie terrestre y el meridiano que se toma como 0 (es decir, el meridiano de Greenwich. Su valor oscila entre 0 y 360 (este último valor corresponde de nuevo a 0).
\item[TestCiudades] Qué tiene que realizar lo siguiente:
\begin{itemize}
\item Contener el método main.
\item Crear un objeto \emph{Pais} denominado \emph{ESPAÑA	}.
\item Mediante la clase \emph{Scanner} leer los datos para crear tantos objetos \emph{Ciudad} como sea posible, usando los datos aportados al final del documento. Observa que hay valores de latitud y longitud que harán saltar las correspondientes excepciones.
\item Usando programacion defensiva mediante expresiones regulares, no permitas datos que no sean cadenas para el nombre de ciudad, ni numeros sin formato correcto para la longitud y latitud, y que no sean enteros para el numero de habitantes.
\item Añadir los objetos \emph{Ciudad} válidos al objeto \emph{Pais}
\item Mostrar en consola la referencia del objeto \emph{Pais} de acuerdo a la sobreescritura del método \emph{toString}
\item Realizar la llamada al método \emph{obtenerDatosDePais} que se detalla abajo.
\item Crear un método aparte del main, denominado \emph{obtenerDatosDePais}, que tenga como argumento eun objeto \emph{Pais} y que muestre en consola las llamadas del  método de la clase \emph{Pais} que nos devuelve el objeto \emph{Ciudad} con mas habitantes, mostrando el nombre de la ciudad y el número de habitantes, así como la llamada del método que nos devuelve la población media, usando \emph{printf} formateando con dos decimales
\end{itemize}
\end{description}
Consideraciones a tener en cuenta a la hora de hacer el ejercicio:
Criterios de evaluación:
\begin{description}
\item[2 pto.] Por el control correcto de las excepciones.
\item[1 pto.] Por la clase \emph{Ciudad}
\item[2.5 ptos.] Por la clase \emph{Pais}
\item[2.5 ptos.] Por la clase \emph{TestCiudades}
\end{description}
\textbf{Formato del fichero de subida:} se subirá el \emph{workspace} del proyecto del examen.
\end{questions}
\begin{verbatim}
     Ciudad     Latitud      Longitud             Habitantes
     Madrid     40.461688    -3.70319             3273049
     Jaén       37.7793464   -3.78515             114238
     Barcelona  41.3850379    2.17336             1608746
     Alicante   38.34601,    -0.49069             329988
     Error1     120	          0                   12222
     Error2     12	           500                 600
     Lugo       43.0099289    -7.556834900000013  97995
\end{verbatim}
\end{document}
