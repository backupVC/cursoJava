\documentclass[addpoints,12pt]{exam}
\usepackage[spanish]{babel}
%\usepackage[latin1]{inputenc}
\usepackage[utf8x]{inputenc}
\pagestyle{empty}
\begin{document}
\begin{center}
\fbox{\fbox{\parbox{5.5in}{\centering {\LARGE EXAMEN TIPO A}\\El ejercicio se realizará en eclipse. Usa un workspace distinto al que usas de forma habitual. Crea un proyecto de java nuevo denominado Examen.}}}
\end{center}
\vspace{0.1in}
%\makebox[\textwidth]{Nombre:\enspace\hrulefill}
\begin{questions}
\question Queremos hacer un programa para registrar los datos de una pelicula, para esto desarrollaremos las siguientes clase con las correspondientes métodos:
\begin{description}
\item[clase Actor] Qué tiene los siguientes atributos y métodos:
\begin{itemize}
\item Nombre
\item Apellidos
\item Edad
\item Un constructor que inicialice dichos atributos.
\item Los correspondientes \emph{getters}
\item Sobreescribe el método \emph{toString()} para que cuando tratemos la referencia de estos objetos como \emph{String}, nos muestre al actor con el siguiente formato:\\
Nombre: nombre\_actor - apellidos\_actor --- Edad: edad\_actor
\end{itemize}
\item[clase Director] Qué tiene los siguientes atributos y métodos:
\begin{itemize}
\item Nombre.
\item Apellidos.
\item Un constructor que inicialice dichos atributos.
\item Los correspondientes \emph{getters}
\item Sobreescribe el método \emph{toString()} para que cuando tratemos la referencia de estos objetos como \emph{String}, nos muestre al director con el siguiente formato:\\
\emph{Nombre: nombre\_director - apellidos\_director}
\end{itemize}
\item[Película] Qué tiene los siguientes atributos y métodos:
\begin{itemize}
\item Una colecion de actores
\item Una colección de directores.
\item Presupuesto.
\item Dos métodos que añadan actores  a la colección de actores y otro que haga lo mismo con los directores.
\item Un método que defina el presupuesto de la película. (\emph{setter})
\item Un método que devuelva la edad media de los actores que conforman la película.
\item Los correspondientes \emph{\emph{getters}}
\item No uses un constructor en esta clase.
\item Sobreescribe el método \emph{toString()} para que cuando mostramos la película como \emph{String} nos muestre los datos con el siguiente formato:\\
\newpage
\emph{ACTORES:\\
lista de actores \\
DIRECTORES:\\
lista de directores\\
PRESUPUESTO\\
presupuesto en euros}
\end{itemize}
\item[Clase TestPelicula] Completa el fichero \emph{TestPelicula.java} que se aporta.
\end{description}
Consideraciones a tener en cuenta a la hora de hacer el ejercicio:
\begin{itemize}
\item Debes implementar las clases anteriores.
\item Completa la clase \emph{TestPelicula.java} de acuerdo a las instrucciones que incorpora dicho fichero.
\item Deberas entregar el proyecto de la siguiente manera:
\begin{enumerate}
\item El directorio \textbf{src} del proyecto.
\item El proyecto completo de eclipse. 
\end{enumerate}
\end{itemize}
Criterios de evaluación:
\begin{description}
\item[1.5 pto.] Por el archivo Actor.java funcionando correctamente.
\item[1.5 pto.] Por el archivo Director.java funcionando correctamente.
\item[3.5 pto.] Por el archivo Pelicula.java funcionando correctamente.
\item[3.5 ptos.] Por TestPelicula.java funcionando correctamente.
\end{description}
\textbf{Formato del fichero de subida:} se subirá un archivo comprimido de la siguiente forma \emph{nombreApellidos.tar.gz} o \emph{nombre.Apellidos.zip}. incluyendo tanto el proyecto de eclipse como la carpeta \emph{src}.
\end{questions}
\end{document}
