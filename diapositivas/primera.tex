\documentclass{beamer} 
\usetheme{default} 
\usecolortheme{albatross}
\setbeamercovered{transparent}
\usepackage[spanish]{babel}

%\useoutertheme{umbcfootline} 
%\setbeamertemplate{background canvas}[vertical shading][bottom=red!20,top=yellow!30] 


%\usepackage[spanish]{babel}
%\usepackage[latin1]{inputenc}
\usepackage[utf8x]{inputenc}
\usepackage{multicol}


\title{MÓDULO PROGRAMACIÓN}

\author{Manuel J. Molino Milla}

%\date{\today} %

\institute{IES Virgen del Carmen \and Departamento de Informática}




\beamerdefaultoverlayspecification{<+->}

\begin{document}


\begin{frame}
  \titlepage
\end{frame}





\section{INTRODUCCION}


\begin{frame}
\frametitle{Introducción}


\begin{itemize}[<+-| alert@+>]
\item Profesor titular del módulo: Manuel Molino.
\item Profesor desdoble: Luis Molina
\item Desdobles: 3
\item Módulo de duración: 256 horas.
\item 8 horas semanales.
\item Módulo presencial.
\item Distribución horaria es: 2+2+2+2
\end{itemize}
\pause

\end{frame}

\begin{frame}
\frametitle{Sistemas Operativos y Lenguajes de programación}
\begin{itemize}[<+-| alert@+>]
\item El ciclo se denomina desarrollo de aplicaciones multiplataforma.
\item Implica que podemos trabajar en cualquier sistema operativo.
\item En nuestro caso podremos trabajar en Ubuntu o Windows.
\item El lenguaje de programación a estudiar será \emph{Java}
\item En cuanto al IDE (entorno de programación) será \emph{IntelliJ IDEA}
\item Se puede usar otro IDE como \emph{Eclipse, Netbeans}
\item También se pueden usar otros como \emph{atom, visual code, sublimetext, \dots}
\end{itemize}
\pause
\end{frame}

\section{CONTENIDOS}

\begin{frame}[fragile]
\frametitle{Contenidos del decreto}
\begin{itemize}[<+->]
\item Identificación de los elementos de un programa informático.
\item Utilización de objetos.
\item Uso de estructuras de control.
\item Lectura y escritura de información.
\item Aplicación de las estructuras de almacenamiento.
\item Utilización avanzada de clases.
\item Mantenimiento de la persistencia de los objetos.
\item Gestión de bases de datos relacionales.
\end{itemize}
\end{frame}

\begin{frame}
\frametitle{Primera Evaluación} 
\begin{itemize}[<+-| alert@+>]
\item Introducción a la programación.
\item Introducción a Java.
\item Introducción a sistema de control de versiones (git/gitHub) (opcional)
\item Elementos de un programa informático: variables, tipos de datos, literales, constantes, operaciones, \dots
\item Estructuras de control.
\item Introducción a la programación orientad a objetos (POO).
\item Introducción a colecciones en java.
\item Packages.
\item Entorno de programación.
\end{itemize}
\pause
\end{frame}

\begin{frame}[fragile]
\frametitle{Segunda evaluación}
\begin{itemize}[<+-| alert@+>]
\item Lectura y escritura de información. Operaciones de E/S.
\item XML.
\item Depuración.
\item TDD 
\item Ampliación de colecciones en Java.
\item Excepciones.
\item Conceptos avanzados en POO: herencia y polimorfismo.
\end{itemize}
\pause
\end{frame}

\begin{frame}[fragile]
\frametitle{Tercera evaluación}
\begin{itemize}[<+-| alert@+>]
\item Bases de datos relacionales: \emph{sqlite} 
\item Acceso a base de datos con Java: \emph{JDBC}
\item Desarrollo de interfaces con \emph{swing}.
\item Base de datos orientadas a objetos.
\end{itemize}
\pause
\end{frame}



\section{METODOLOGÍA}
\begin{frame}[fragile]
\frametitle{Metodología}
\begin{itemize}[<+->]
\item Explicación de los correspondientes temas.
\item Elaboración de programas acorde a la materia impartida.
\item Realización de ejercicios.
\item La corrección de dichos ejercicios quedarán en un repositorio de \emph{gitHub}
\item También quedarán en el repositorio todos los programas que elaboremos en el módulo.
\end{itemize}
\end{frame}

\section{CRITERIOS EVALUACIÓN}
\begin{frame}
\frametitle{Herramientas evaluación}
\begin{itemize}[<+-| alert@+>]
\item Exámenes en el que se evalúan tanto aspectos teóricos como prácticos.
\item La teoría se evalua con un examen de tipo test y promedia con un 15\% la nota del examen.
\item La parte práctica consistirá en el desarrollo de un programa que abarque la materia estudiada, al ser evaluación continua cualquier tema podrá ser exigido, aunque se haga mas énfasis en la parte impartida en dicho trimestre. Dicha parte práctica pondera con un 85\% de la nota del examen.
\item Realización de un proyecto de programación.
\end{itemize}
\pause
\end{frame}




\begin{frame}
\frametitle{Criterios evaluación}
\begin{itemize}[<+-| alert@+>]
\item Se aprueba el módulo siempre y cuando la nota final del mismo se cinco o mas.
\item La nota final se calcula con el siguiente criterio: 80\% notas de exámenes +  20\% nota del proyecto.
%\item La nota final se calcula con el siguiente criterio: 90\% notas de exámenes +  10\% actitud. (En el caso de que no se realicen las prácticas).
%\item En el caso de la opcionalidad de las prácticas, se establecerá como nota final, la mayor nota de las dos fórmulas anteriores.
%\item En el caso de que se establezcan el carácter obligatorio de las prácticas, es indispensable haber presentado las mismas y haber sido evaluadas con un cinco o mas cada una de ellas, para aprobar el módulo.
%\item También es indispensable haber obtenido una calificación de cinco puntos o mas en los exámenes.
\item Pruebas de recuperación no existen. 
\end{itemize}
\pause
\end{frame}




\begin{frame}
\frametitle{Proceso de recuperación}
\begin{itemize}[<+-| alert@+>]
\item En el caso que el alumno no haya superado el módulo satisfactoriamente en mayo, hay un periodo de recuperación en junio.
%\item El alumno podrá presentar las prácticas no realizadas o suspensas en este periodo, en el caso de que sean obligatorias.
\item El criterio de evaluación será igual que en el anterior periodo.
\item El alumno podrá subir la califiación en el periodo de recuperación, aunque haya aprobado en la tercera evaluación.
\end{itemize}
\pause
\end{frame}



\section{AULAS}

\begin{frame}
\frametitle{Normas}

\begin{itemize}[<+-| alert@+>]
\item Los equipos de trabajo son compartidos con los alumnos de primero de ASIR.
\item Queda prohibido entrar en cuentas de usuario que no sean la dispuestas para el alumno para el desarrollo normal de las clases.
\item No se puede instalar software sin el permiso de profesor.
\item No se hacen descansos entre las franjas horarias del módulo.
\item Subid una fotografia a la plataforma.
\end{itemize}
\pause
\end{frame}


\begin{frame}
\frametitle{Preguntas} 
\begin{figure}
\includegraphics[scale=0.9]{tercera_evaluacion/jdbc/imagenes/dudas.png} 
\end{figure} 
\end{frame}


\end{document}

