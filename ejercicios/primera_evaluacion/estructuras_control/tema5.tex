\documentclass[4paper]{article}
\usepackage[spanish]{babel}
%\usepackage[ansinew]{inputenc}
\usepackage[utf8x]{inputenc}
%\usepackage[T1]{fontenc}
\usepackage{graphicx}
\usepackage{multicol}
\renewcommand{\tablename}{Tabla}
\author{Manuel Molino Milla \and Luis Molina Garzón}
\title{\textbf{Programación}
\\ESTRUCTURAS DE CONTROL}
\date{\today}

\begin{document}
\maketitle
\tableofcontents
%\setlongtables
%\section{API de Java}
%\subsection{Ejercicio 1}
%La API String de Java, contiene entre otros, los métodos abajo indicados en la tabla. Rellena la información correspondiente:
%\par
%\vspace{0,5cm}
%\begin{tabular}{|c|c|c|c|c|}
%\hline
%\bf{nombre método} & \bf{valor retorno} & \bf{parámetros} & \bf{Breve descripción} & \bf{Ejemplo} \\
%\hline
%contains &  &  & &  \\
%\hline
%endsWith &  &  & &  \\
%\hline
%equalsIgnoreCase &  &  & &  \\
%\hline
%isEmpty &  &  & &  \\
%\hline
%startsWith &  &  & &  \\
%\hline
%charAt &  &  & &  \\
%\hline
%length &  &  & &  \\
%\hline
%replace &  &  & &  \\
%\hline
%toUpperCase &  &  & &  \\
%\hline
%toLowerCase &  &  & &  \\
%\hline
%substring &  &  & &  \\
%\hline
%\end{tabular}


\section{Estructuras de control}
\subsection{Ejercicio 1}
En una clase denominada \emph{Numeros} implementa métodos con las siguientes especificaciones:
\begin{itemize}
\item Muestre en consola los números del 1 al 10.
\item Muestre en consola la siguiente serie: 20 25 30 35 \dots 70 75 80
\item Devuelva la siguiente serie: 100 98 96 94 \dots 56 54 52 50
\item Devuelva la suma de los números enteros del 1 a N, siendo N un número entero y el parámetro que se pasa al método. El método debe devolver un \emph{int}
\item Devuelva la suma de los cuadrados de los N primeros números naturales. El método debe devolver un \emph{int}
\item Muestre en consola independientemente la suma de los pares e impares comprendidos entre 1 y N
\end{itemize}
Realiza el ejercicio usando una vez el bucle for, otra vez el bucle while y también debes usar la estructura do-while


%\subsection{Ejercicio 2 -OPCIONAL-}
%En español hay dos maneras para formar el plural de los sustantivos y adjetivos: -s y -es. Las reglas en las que se basan son:
% \begin{itemize}
%\item \textbf{Sustantivos y adjetivos terminados en vocal átona o en -e tónica}. Forman el plural con -s: \emph{casas, estudiantes, taxis, planos, tribus, comités.}
%\item \textbf{Sustantivos y adjetivos terminados en -a o en -o tónicas}.Forman el plural  con -s: \emph{papás, sofás, bajás, burós, rococós, dominós.}
%\item  \textbf{Sustantivos y adjetivos terminados en -i o en -u tónicas}. Admiten generalmente dos formas de plural, una con -es y otra con -s, aunque en la lengua culta suele preferirse la primera: \emph{bisturíes o bisturís, carmesíes o carmesís, tisúes o tisús, tabúes o tabús}.
%\item \textbf{Sustantivos y adjetivos terminados en -y precedida de vocal}. Forman tradicionalmente su plural con -es: \emph{rey, pl. reyes; ley, pl. leyes; buey, pl. bueyes; ay, pl. ayes; convoy, pl. convoyes; bocoy, pl. bocoyes}
%\item  \textbf{Sustantivos y adjetivos terminados en -s o en -x.} Permanecen invariables: \emph{crisis, pl. crisis; tórax, pl. tórax; fórceps, pl. fórceps}.
%\item \textbf{Sustantivos y adjetivos terminados en -l, -r, -n, -d, -z, -j.} Forman el plural con -es: \emph{dócil, pl. dóciles; color, pl. colores; pan, pl. panes; césped, pl. céspedes; cáliz, pl. cálices; reloj, pl. relojes}. 
%\end{itemize}
%Crea una clase llamada \emph{Plural} que tenga un método que devuelva el plural de dicha palabra.\\
%Comprueba el funciomaniento de dicho método en otra clase que aporte el método \emph{main}
%\par
%\vspace{0.2cm}
%{\Large Ayuda}: consulta el API de la clase \emph{String}

\section{APIs de Java}

\subsection{Ejercicio 1}
Realiza un programa en Java, que introduzca como argumentos el nombre y apellidos de alguién y posteriormente muestre el siguiente mensaje:
\begin{quote}
Hola nombre apellidos
\end{quote}
\subsection{Ejercicio 2}
Busca información sobre la clase Scanner, usada para leer datos desde el teclado o desde un fichero y explica que realizan los siguientes metodos:
\begin{itemize}
\item Scanner(System.in)
\item hasNext()
\item hasNextInt()
\item hasNextDouble()
\item next()
\item nextInt()
\item nextDouble()
\end{itemize}

\subsection{Ejercicio 3}
Realiza un programa en Java, que lea desde le entrada estandar tu nombre y apellidos y posteriomente presente por pantalla el siguiente mensaje: 
\begin{quote}
Hola nombre apellidos
\end{quote}




\subsection{Ejercicio 4}
Realiza un programa en Java, que lea desde la entrada un número entero de tres cifras y posteriormente muestre sus cifras por separado. Haz todo en un metodo \emph{main} de una clase denominadas Cifras. Ejemplo:
\begin{quote}
Número 457\\
Cifra1 4\\
Cifra2 5\\
Cifra3 7
\end{quote}

\subsection{Ejercicio 5}
Programa que se le pase como argumentos tres números enteros H, M, S que contienen hora, minutos y segundos respectivamente. Comprueba si la hora que indican es una hora válida. Ejemplo de ejecución:
\begin{quote}
java Tiempo 22 10 15\\
Hora válida\\
java Tiempo 25 10 15\\
Hora no válida\\
\end{quote}

\subsection{Ejercicio }
Programa que solicite un número y un número de columnas, de manera que imprima desde el 1 al numero solicitado, en en número de columnas pedido. Ejemplo:
\begin{verbatim}
Introduce nº de columnas
3
Introduce un nº
15
    1    2    3
    4    5    6
    7    8    9
   10   11   12
   13   14   15
\end{verbatim}
Otro ejemplo:
\begin{verbatim}
Introduce nº de columnas
7
Introduce un nº
100
    1    2    3    4    5    6    7
    8    9   10   11   12   13   14
   15   16   17   18   19   20   21
   22   23   24   25   26   27   28
   29   30   31   32   33   34   35
   36   37   38   39   40   41   42
   43   44   45   46   47   48   49
   50   51   52   53   54   55   56
   57   58   59   60   61   62   63
   64   65   66   67   68   69   70
   71   72   73   74   75   76   77
   78   79   80   81   82   83   84
   85   86   87   88   89   90   91
   92   93   94   95   96   97   98
   99  100
\end{verbatim}


\subsection{Ejercicio 6}
Los sistemas de ecuaciones lineales, en el caso de dos ecuaciones con dos incógnitas se pueden representar de forma genérica de la siguiente forma:
\begin{large}
\begin{center}
$\left.
a \cdot x + b \cdot y  = e \atop
c \cdot x + d \cdot y = f
\right\}$
\end{center}
\end{large}
Un sistema de ecuaciones tiene solución si:
\begin{center}
$a \cdot d - b \cdot c  \not= 0$
\end{center}
La solución de un sistema de dos ecuaciones con dos incognitas, en el caso que lo tenga, viene dado por la siguiente regla, derivada de la regla de Cramer:
\begin{large}
\begin{center}
$x = \frac{e \cdot d - b \cdot f}{a \cdot d - b \cdot c}$ \\
\vspace*{0.2cm}
$y = \frac{a \cdot f - e \cdot c}{a \cdot d - b \cdot c}$
\end{center}
\end{large}
Se quiere realizar una clase denomina \emph{Ecuacion} que contenga dos métodos:
\begin{itemize}
\item Un método boolean que devuelva cierto o falso si el sistema es resoluble o no.
\item Dos métodos que devuelvan el valor de \emph{x} e \emph{y}. Usa \emph{double} como tipos, al menos en los valores de \emph{x} e \emph{y}
\item Los argumentos en los métodos serán los coeficientes que acompañan a las incognitas, además del término independiente \emph{(a, b, c, \dots)}
\end{itemize}
Crea una clase TestEcuación, con el método \emph{main} y que resuelva las siguientes ecuaciones:
\begin{large}
\begin{center}
$\left.
x + y  = 1 \atop
2 \cdot x + 2 \cdot y = 2
\right\}$
\end{center}
\end{large}
\begin{large}
\begin{center}
$\left.
2 \cdot x + y  = 7 \atop
-x + 2 \cdot y = -1
\right\}$
\end{center}
\end{large}
Debe indicar por pantalla, si el sistema es resoluble o no. Y en el caso que sea resoluble debe mostrar la resolucion de dicho sistema de ecuaciones.

\end{document}
