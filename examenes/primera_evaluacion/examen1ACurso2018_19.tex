\documentclass[addpoints,12pt]{exam}
\usepackage[spanish]{babel}
%\usepackage[latin1]{inputenc}
\usepackage[utf8x]{inputenc}
\pagestyle{empty}
\begin{document}
\begin{center}
\fbox{\fbox{\parbox{5.5in}{\centering {\LARGE EXAMEN TIPO A}\\Sigue las intrucciones que marca cada ejercicio, no solo para la realización del mismo sino también para la entrega de los mismos.\\\emph{No se permite ningun documento de ayuda, excepto los aportados por el profesor}\\\emph{No se corregirá ningun ejercicio que tenga errores de compilación}}}}
\end{center}
\vspace{0.1in}
%\makebox[\textwidth]{Nombre:\enspace\hrulefill}
Realiza los siguientes programas en \emph{Java}
\begin{questions}
\question(8 ptos) Los tres lados \emph{a, b y c} de un triángulo deben satisfacer la desigualdad triangular: cada uno de los lados no puede ser más largo que la suma de los otros dos. Realiza una clase denominada \emph{Triangulo} que contenga los siguientes métodos:
\begin{itemize}
\item Un método estático que recibe como parámetros los tres lados del triángulo y nos diga si dicho triángulo es posible de acuerdo al principio de desigualdad triangular antes mencionado. Hay que hacer tres comprobaciones, una por cada lado.
\item Otro método estático, que dado los tres lados nos devuelva el perímetro de dicho triángulo.
\item Otro método, también estático, que nos diga si es un triángulo equilatero, esto será cierto cuando los tres lados sean iguales.
\item Otro método que nos diga si es un triángulo rectángulo. Un triángulo rectángulo satisface el teorema de pitágoras ($a^2 = b^2 + c^2$). Para comprobar dicho teorema, previamente tienes que determinar cual es el lado mas grande comparando todos los lados y ese será la posible hipotenusa \emph{(a en la fórmula anterior)}.
\item Y por último otro método estático que nos diga si es un triángulo isósoceles, es decir que tiene dos lados iguales y otro desigual.
\end{itemize}
Realiza una clase \emph{TestTriangulo} que lleve acabo las siguientes acciones:
\begin{itemize}
\item Mediante la clase \emph{Scanner} solicita los tres lados del triángulo. Dichos lados deben ser de tipo entero.
\item En el caso que se introduzca un lado con valor negativo o cero, el programa terminará con un mensaje diciendo que dicho lado no se puede admitir.
\item Se llamará al método de la clase \emph{Triangulo} que ratifica la desigualdad triangular, de manera que si el triángulo no es posible indicará que dicho triángulo no es válido y terminará el programa.
\item En caso de que el triángulo es viable, el programa dirá si es o no equilatero, si es o no rectángulo y si es o no isósceles.
\end{itemize}
\newpage
Puedes probar los siguientes valores:
\begin{itemize}
\item 1 5 9 No es posible el triángulo
\item -1 2 3 No es válido algún lado
\item 1 1 1 Triángulo equilatero, no rectángulo y no isósceles.
\item 3 4 5 Triángulo no equilatero, rectángulo y no isósceles.
\item 5 5 7 Triángulo no equilatero, no rectángulo y sí isósceles.
\item 6 7 8 Triángulo no equilatero, no rectángulo y no isósceles.
\end{itemize}
Realiza la documentación de la clase \emph{Triangulo}, incluyendo las etiquetas autor y versión. Recuerda que no solo hay que documentar la cabecera de la clase, sino también todos los métodos.\par
\vspace{0.5cm}
Desglose de la puntuación
\begin{parts}
\part (4 ptos) Se valorará con 1 pto los métodos que comprueba la desigualdad triangular, si es isósceles y si es rectángulo. Los otros dos métodos se valoran con 0.5 ptos.
\part (3 ptos) Si implementas correctamente la clase \emph{TestTriangulo}, se tendrá en cuenta el uso correcto de la clase \emph{Scanner}, la llamada a los métodos de la clase \emph{Triangulo}, el uso correcto de estructura de control para realizar las llamadas de los métodos inlcuidos la llamada al método que calcula el perímetro del triágulo y las salidas a consola de los correspondientes mensajes valorando el uso de \emph{printf}
\part (1 pto) Por la documentación de la clase \emph{Triangulo}. 

\end{parts}
En el caso que no sepas modularizar en dos clases, usa una única clase -\emph{TestTriangulo}- que incluya el método \emph{main}. En este caso la puntuación máxima obtenida en el ejercicio será de 6 puntos. En este caso no merece la pena que realices documentación-
\end{questions}

\vspace{0,5cm}
Debes entregar los siguientes documentos:
\begin{itemize}
\item Triangulo.java
\item TestTriangulo.java
\end{itemize}

Entrégalos en un documento comprimido con el formato:\emph{ apellidosNombre.tar.gz} o \emph{apellidosNombre.zip}. Tanto a la plataforma \emph{moodle} como al profesor\par
\vspace{0.5cm}
La puntuación es de 8 puntos, pues la teoría se valora con 2 puntos.
\end{document}
