\documentclass[addpoints,12pt]{exam}
\usepackage[spanish]{babel}
%\usepackage[latin1]{inputenc}
\usepackage[utf8x]{inputenc}
\pagestyle{empty}
\begin{document}
\begin{center}
\fbox{\fbox{\parbox{5.5in}{\centering {\LARGE EXAMEN TIPO A}\\Sigue las intrucciones que marca cada ejercicio, no solo para la realización del mismo sino también para la entrega de los mismos.\\\emph{No se permite ningun documento de ayuda, excepto los aportados por el profesor}}}}
\end{center}
\vspace{0.1in}
%\makebox[\textwidth]{Nombre:\enspace\hrulefill}
\begin{questions}
\question(4 ptos)  Los sistemas de ecuaciones lineales, en el caso de dos ecuaciones con dos incógnitas se pueden representar de forma genérica de la siguiente forma:
\begin{large}
\begin{center}
$\left.
a \cdot x + b \cdot y  = e \atop
c \cdot x + d \cdot y = f
\right\}$
\end{center}
\end{large}
Un sistema de ecuaciones tiene solución si:
\begin{center}
$a \cdot d - b \cdot c  \not= 0$
\end{center}
La solución de un sistema de dos ecuaciones con dos incognitas, en el caso que lo tenga, viene dado por la siguiente regla, derivada de la regla de Cramer:
\begin{large}
\begin{center}
$x = \frac{e \cdot d - b \cdot f}{a \cdot d - b \cdot c}$ \\
\vspace*{0.2cm}
$y = \frac{a \cdot f - e \cdot c}{a \cdot d - b \cdot c}$
\end{center}
\end{large}
Se quiere realizar una clase que represente a dicho sistemas de dos ecuaciones con dos incógnitas, para esto ten en cuenta lo siguiente:
\begin{itemize}
\item Usa como atributos los que consideres oportunos, de acuerdo a la representación genérica de un sistema de dos ecuaciones con dos incógnitas, y los tipos que sean \emph{double}
\item Un constructor para los objetos de tipo Ecuación.
\item Un método boolean que devuelva cierto o falso si el sistema es resoluble o no.
\item Dos métodos que devuelvan el valor de x e y. Los tipos a devolver deben ser acordes con los datos de inicialización.
\end{itemize}
\begin{parts}
\part
(2 ptos) Crea una clase TestEcuación, con el método \emph{main} que cree los dos siguientes objetos:
\begin{large}
\begin{center}
$\left.
x + y  = 1 \atop
2 \cdot x + 2 \cdot y = 2
\right\}$
\end{center}
\end{large}
\begin{large}
\begin{center}
$\left.
2 \cdot x + y  = 7 \atop
-x + 2 \cdot y = -1
\right\}$
\end{center}
\end{large}
Debe indicar por pantalla, si el sistema es resoluble o no. Y en el caso que sea resoluble debe mostrar la resolucion de dicho sistema de ecuaciones.
\part
(2 ptos) Modifica el archivo \emph{build.xml} que se suministra para que tenga los siguientes objetivos:
\begin{description}
\item[compila] Que solo compile el proyecto donde las fuentes se encuentren en un directorio llamado \emph{src} y las clases se disponga en un directorio denominado \emph{bin}
\item[ejecuta] Que compile y ejecute el proyecto, tienendo en cueta la dependencia anterior, puedes ejecutarlo como empaquetado.
\item[documenta] Que cree en un directorio denominado \emph{doc} donde se incluya la documentación de javadoc
\end{description}
\end{parts}
\question(3 ptos) Segundo ejercicio.
\begin{parts}

\part Con las plantillas que se te suministra, añade el código necesario para que puedas realizar un programa que dada una palabra nos diga:

\begin{itemize}
\item Si empieza y acaba por la misma letra, independientemente que sea la primera mayúscula y la última minúscula.
\item Si contiene dentro de la palabra (no al final) la misma letra con la que empieza, independientemente que sean minúsculas o mayusculas algunas de ellas o todas.
\end{itemize}
\part
Repite el ejercicio, tanto la clase como el test para que se adjuste al paradigma de POO
\end{parts}
\end{questions}
\vspace{0,5cm}
\hrule
\vspace{0,5cm}
Entrega del ejercicio, el ejercicio se entregará en un documento comprimido con el formato: apellidosNombre.tar.gz o apellidosNombre.zip con la siguiente estrucutura:
\begin{itemize}
\item Se crearán dos directorios, uno denominado \emph{ejercicio1} y otro \emph{ejercicio2}.
\item En el ejercicio1 si no se realiza la parte del \emph{ant} se entregaran tantos los documentos \emph{.class} y \emph{.java}
\item Si en el ejercicio1 se ha realizado la parte del \emph{ant} se creara el subdirectorio \emph{src} donde se entregara las fuentes del ejercicio, no se entregara \emph{NADA} mas, excepto el fichero \emph{build.xml} modificado en la raíz del subdirectorio \emph{ejercicio1}
\item En el ejercicio2 se crearan dos subdirectorios denominados \emph{sinobjetos} y otro \emph{objeto} donde se pondran tanto las fuentes como los ejecutables en cada uno de ellos.
\end{itemize}
\end{document}
