\documentclass[4paper]{article}
\usepackage[spanish]{babel}
%\usepackage[ansinew]{inputenc}
\usepackage[utf8x]{inputenc}
%\usepackage[utf-8]{inputenc}
%\usepackage[T1]{fontenc}
\usepackage{graphicx}
\usepackage{multicol}
%\usepackage{longtable}
%\usepackage{array}
%\usepackage{multirow}
%\usepackage[latin1]{inputenc}
%\inputencoding{latin1}
%\usepackage{eurosym}

\renewcommand{\tablename}{Tabla}
\renewcommand{\S}{Java}
\author{Examen tercera evaluación}
\author{Tipo A}
\title{\textbf{\S}}
\date{\today}

\begin{document}
\maketitle 
%\tableofcontents
%\newpage
\section*{Descripción del ejercicio}
El ejercicio consiste en la creación una aplicación para gestionar el stock de coches de una empresa de compra venta. El registro de cada coche viene determinado por los siguientes aspectos
\begin{description}
\item[Codigo] que una secuencia de 17 caracteres alfanumericos.
\item[Marca] que es la marca de dicho coche.
\item[Modelo] que corresponde al modelo de dicho coche.
\item[Año de fabricacion] correspondiente al año de fabricacion de dicho coche.
\end{description}
Para ver datos y tipos se aporta la base de datos \emph{coches.db} realizada en \emph{sqlite}

\section*{Creacion de la aplicación}
Se aporta el esquema de un proyecto de eclipse, con la definición de la mayor parte de clases y metodos a implementar.\\
Tenemos que crear una conexión a la base de datos usando el patron de instancia única o \emph{singlenton} \\
Debemos crear una aplicación en \emph{Java} que nos permita hacer operaciones \emph{CRUD} sobre la \emph{base de datos} anterior, las operaciones que debes realizar son:
\begin{itemize}
\item Listar todos los coches.
\item Listar los coches cuyo año de fabricación este entre dos años dados.
\item Borrar un coche dado el código del mismo.
\item Actualizar el modelo de un coche.
\item Crear un coche  nuevo.
\end{itemize}
Para esto debemos usar los patrones \emph{DAO} y \emph{DTO}. Se aporta la interfaz \emph{DAO} para su posterior implementación. En el caso de la clase que usa el patron \emph{DAO} sobreescribe los metodos \emph{toString()}, \emph{hasCode() y equals()}, en este caso dos objetos \emph{CocheDTO} son iguales cuando tengan el mismo codigo.
\par 
Tambien se debe implementar una clase \emph{Logs} que nos ayude a recoger las operaciones realizadas sobre la base de datos, de manera que podamos guardar en un fichero la fecha de realizacion de dicha operación, usando el método \emph{now} de la clase \emph{LocalDateTime}, mas la operacién realizada \emph{(lectura, borrado, actualización o creación de datos)}. Los datos \textbf{no deben sobreescribir} a lo que ya tenga el fichero.\par
Aparte se aporta la clase \emph{Principal} donde arranca la aplicación y se detalla las operaciones a realizar.

\section*{Diagramas}
Se realizara los diagramas UML de las clase que generan la conexion a la base de datos, las relacionadas con los patrones DAO y DTO, y finalmente la clase \emph{Logs}.

\section*{Aspectos a tener en cuenta}
Se usaran los paradigma de \emph{POO} y el tratamiento de excepciones se hara de acuerdo a las especificaciones de \emph{java 1.7}. Todos los flujos que se abran deben cerrarse en la aplicacion, desde la apertura de la conexion a la base de datos, operaciones sobre ella, lectura y escrituras \emph{I/O}.\par
La lectura y escritura sobre ficheros se hara usando técnicas de \emph{buffer}\\
No olvides cerrar la conexión a la base de datos correctamente.\\
La introducción de parámetros en las acciones sobre la base de datos deben evitar inyecctión SQL.


\section*{Criterios de evaluacion}
Los criterios de evaluación son los indicados a continuación:\par 
\vspace*{0.5cm}
\begin{tabular}{|c|c|}
\hline
\textbf{CRITERIO EVALUACION} & \textbf{PUNTUACION} \\
\hline
Programa funcionado correctamente & 1 pto\\
\hline
Clase \emph{Log} & 1.5 ptos.\\
\hline
Clase \emph{DAO} & 1.5 ptos.\\
\hline
Clase \emph{DTO} & 3 ptos.\\
\hline
Clase \emph{Main} & 2 ptos.\\
\hline
Diagrama UML & 1 pto.\\
\hline

\end{tabular}
\par 
\vspace{0.3cm}
De forma adicicional se concedera un punto en el caso que se prepare una interfaz grafica para recepcionar los datos pedidios en la clase \emph{Main}
\section*{Subida de ficheros}
Debes subir el \emph{workspace} del proyecto del examen.
\end{document}
