\documentclass[a4paper,spanish]{article}
\usepackage[spanish]{babel}
\usepackage[ansinew]{inputenc}
\usepackage[T1]{fontenc}
\usepackage{graphicx}
\usepackage{multicol}
\usepackage{longtable}
\usepackage{array}
\usepackage{multirow}

\renewcommand{\tablename}{Tabla}
\renewcommand{\S}{Base de datos y Java}
\author{Examen tercera evaluaci�n}
\title{\textbf{\S}}
\date{\today}

\begin{document}
\maketitle 
%\tableofcontents
%\newpage
\section*{Creacion de la BD}
El ejercicio consiste en la creaci�n de una base de datos llamada Mundo. La debes crear en \emph{sqlite} y sin integridad referencial, es decir sin claves foraneas. Las tablas a contemplar son las siguientes:
\begin{description}
\item[paises] cuya estructura esta presente en el fichero \emph{paises.sql} que tambien incluye los datos.
\item[continentes] que contiene un \emph{id} de tipo \emph{INTEGER} que es la clave primaria mas un campo de tipo \emph{TEXT} que corresponde al nombre del continente. El valor de los datos debes rellenarlo como tu quieras con el nombre de los continentes.
\item[contiene] que corresponde a la relacion de las entidades de ambas tablas, incluyendo �nicamente las llaves primarias de las tablas anteriores. Rellena diez datos que contemplen paises en todos los continentes, debes incluir a \emph{Spain}
\end{description}
Crea un trigger que nos registre los paises borrados en una nueva tabla, solo tienes que registrar el nombre del pa�s.\par 
Una vez creada la tabla con los datos, crea una copia de seguridad denominada \emph{Mundo.back} que deber�s entregar a la hora de acabar el examen. 

\section*{Creacion de clases}
Crea las clases siguientes usando \emph{jdbc}. En cada clase se conecta y se desconecta de la BD.
\begin{itemize}
\item Una clase por cada tabla con sus datos correspondientes exceptuando la relaci�n.
\item Una clase que facilite la consulta del numero total de paises.
\item Una clase que imprima el nombre de todos los paises que empiezan por \emph{C}
\item Una clase que solicite por la entrada est�ndar el nombre del \emph{id} del pa�s y nos devuelva el nombre del pa�s.
\item Una clase que borre los paises \emph{Spain} y \emph{Portugal} y cree un nuevo pa�s denominado \emph{Iberia} con \emph{id} denominado \emph{IB}. Ten encuenta que las ejecuciones de las sentencia abarcan una transacci�n. Adem�s al no haber integridad referencial debes borrar los datos de la tabla \emph{contiene}
\item Crea una clase denominada TestMundo que contenga el metodo \emph{main} que pruebe todos los metodos anteriormente expuestos.
\end{itemize}

\section*{Criterios de evaluacion}
Los criterios de evaluaci�n son los indicados a continuaci�n:\par 
\vspace*{0.5cm}
\begin{tabular}{|c|c|}
\hline
\textbf{CRITERIO EVALUACION} & \textbf{PUNTUACION} \\
\hline
Creacion de tablas e inserccion de datos & 2 ptos.\\
\hline
Trigguer & 1 pto. \\
\hline
TestMundo.java y resto de java no asociados a la BD & 1 pto.\\
\hline
Clase para conocer el numero de paises  & 1 pto.\\
\hline 
Clase para conocer los paises que empiezan por C & 1 pto.\\
\hline
Clase para conocer el nombre del pais dado el id del mismo & 2 ptos.\\
\hline
Clase conteniendo la transaccion & 2 ptos.\\
\hline
\end{tabular}
\section*{Subida de ficheros}
Debes subir el proyecto de eclipse conteniendo la BD original
\section*{Ayuda}
Para la conexion de la BD en \emph{java} mediante \emph{jdbc} ten encuenta las siguientes sentencias:
\begin{quote}
Class.forName(''org.sqlite.JDBC'');\\
Connection c = DriverManager.getConnection(''jdbc:sqlite:BD'');\\
\end{quote}
Tambi�n puedes consultar las siguientes direcciones web:
\begin{quote}
http://www.sqlite.org/\\
http://docs.oracle.com/javase/tutorial/jdbc/basics/index.html
\end{quote}
\end{document}
