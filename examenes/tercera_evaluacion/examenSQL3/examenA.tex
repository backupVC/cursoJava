\documentclass[4paper]{article}
\usepackage[spanish]{babel}
%\usepackage[ansinew]{inputenc}
\usepackage[utf8x]{inputenc}
%\usepackage[utf-8]{inputenc}
%\usepackage[T1]{fontenc}
\usepackage{graphicx}
\usepackage{multicol}
%\usepackage{longtable}
%\usepackage{array}
%\usepackage{multirow}
%\usepackage[latin1]{inputenc}
%\inputencoding{latin1}

\renewcommand{\tablename}{Tabla}
\renewcommand{\S}{Base de datos y Java}
\author{Examen tercera evaluación}
\author{Tipo A}
\title{\textbf{\S}}
\date{\today}

\begin{document}
\maketitle 
%\tableofcontents
%\newpage
\section*{Creacion de la BD}
El ejercicio consiste en la creación de una base de datos llamada ExamenA, en un SGBD \emph{sqlite} que abarque la siguientes descripción:\par 
Queremos almacenar los datos trabajadores de una empresa en función del departamento en que trabajan. Para esto debemos controlar los siguientes aspectos:
\begin{itemize}
\item Un departamento está formado por uno o mas trabajadores.
\item Los trabajadores pertenecen solamente a un departamento.
\item De los trabajadores interesa conocer su nombre, dni y cualificación (trabajador sin cualificación , cualificado, técnico, administrativo, \dots)
\item Del departamento necesitamos saber su nombre y la localidad donde se encuentra el mismo.
\end{itemize}
Debes aportar el correpondiente diagrama ER, realizar un script para la creación de las correspondientes tablas, que debe contemplar la integridad referencial del \emph{SGBD} e introducir el nombre de diez trabajadores y cinco departamentos. La asignación que hagas de los trabajadores a los departamentos es libre.\par 
Añade al script las siguientes características:
\begin{itemize}
\item Una \emph{vista} que nos de el nombre del trabajador y el nombre del departamento al que pertenece. (Solo esos dos campos).
\item Un \emph{trigger} que controle el historial de departamentos de un trabajador, es decir que cuando un trabajador cambie de departamento, quede registrado el departamento al cuál pertenecía anteriormente y la fecha del cambio.
\end{itemize}

\section*{Creacion de la aplicación}
Debemos crear una aplicación en \emph{Java} que nos permita hacer operaciones sobre la \emph{base de datos} anterior, las operaciones que debes realizar son:
\begin{itemize}
\item Insertar trabajadores, dado sus datos personales y el departamento en el que trabaja.
\item Cambiar a un trabajador de un departamento.
\item Borrar un trabajador.
\item Aprovechando la vista anterior, una consulta que nos de una lista de trabajadores que incluya el departamento donde trabaja.
\end{itemize}
Te aconsejo que crees dos clases \emph{Trabajador} y \emph{Departamento}, con los atributos que mapeen los atributos de ambas tablas en la \emph{BD}, con sus correpondeintes \emph{getters y setter}, constructor, \dots \par 
La \emph{BD} la debes incorporar a la carpeta raíz de tu workspace, de manera que la llamada a la BD la hagas solo con el nombre de dicha \emph{BD} sin indicar nada de ruta o path.
\par 
Comenta el código siempre que te sea posible, no se trata que uses \emph{javadoc}

\section*{Para los que trabajen en Windows}
No usar variables ni comentarios con caracteres que no sean \emph{ASCII}, es decir que sobre todo no utilices acentos ni eñes.

\section*{Criterios de evaluacion}
Los criterios de evaluación son los indicados a continuación:\par 
\vspace*{0.5cm}
\begin{tabular}{|c|c|}
\hline
\textbf{CRITERIO EVALUACION} & \textbf{PUNTUACION} \\
\hline
Creacion de tablas en el script & 1.5 pto.\\
\hline
Integridad referencial en la BD & 0.5 ptos.\\
\hline
Insercción datos & 0.5 ptos.\\
\hline
Creacción de la vista & 0.5 ptos.\\
\hline
Creacción del trigger & 0.5 ptos.\\
\hline
Uso del patrón Singleton para la conexión  & 2 ptos\\
\hline
No uso del patrón en la conexión & 1 ptos.\\
\hline
Uso DAO y DTO aplicación & 3 pto.\\
\hline 
No uso del patrón DAO y DTO pero si POO  & 2 ptos.\\
\hline
No uso del patrón DAO y DTO y sin POO & 1 pto.\\
\hline
Diagrama ER correcto & 0.5 ptos.\\
\hline
Diagrama UML  de las clases& 1 pto.\\
\hline
\end{tabular}
\section*{Subida de ficheros}
Debes incluir tres ficheros:
\begin{itemize}
\item El \emph{workspace} comprimido, en el que esté incluido la \emph{BD}, que quiere decir esto: que incorpores en la carpeta del worspace la \emph{BD} creada en \emph{sqlite}. También debe aparecer el diagrama UML en formato \emph{png o jpeg}.
\item El script de creación de la \emph{BD}.
\item El diagrama ER realizado en \emph{dia} o cualquier otro programa y lo exportas a un formato imagen.
\end{itemize}
\end{document}
