\documentclass[4paper]{article}
\usepackage[spanish]{babel}
%\usepackage[ansinew]{inputenc}
\usepackage[utf8x]{inputenc}
%\usepackage[utf-8]{inputenc}
%\usepackage[T1]{fontenc}
\usepackage{graphicx}
\usepackage{multicol}
%\usepackage{longtable}
%\usepackage{array}
%\usepackage{multirow}
%\usepackage[latin1]{inputenc}
%\inputencoding{latin1}
%\usepackage{eurosym}

\renewcommand{\tablename}{Tabla}
\renewcommand{\S}{Base de datos y Java}
\author{Examen tercera evaluación}
\author{Tipo A}
\title{\textbf{\S}}
\date{\today}

\begin{document}
\maketitle 
%\tableofcontents
%\newpage
\section*{Descripción de la aplicación}
El ejercicio consiste en una aplicación de gestión de datos de personas. Las personas se identifican por los atributos \emph{nombre, trabajo, email y  fecha de nacimiento.}\par 
Los datos se encuentran en un fichero \emph{sql} directo para insertar los mismos, denominado \emph{personas.sql}
\begin{itemize}
\item Crea un proyecto en \emph{eclipse} denominado \emph{ProyectoExamenFinal}
\item Crea un carpeta denominada \emph{BD} que contenga una base de datos denominada \emph{personas.bd} y con una tabla denominada \emph{persona} con los atributos \emph{ nombre, trabajo, email, fechaNacimiento}, realizada en \emph{sqlite}, rellenando los datos con el fichero anterior.
\item Crea una carpeta denominada \emph{ficheros} que contendrá un fichero de texto denominado \emph{menores.txt}
\item Crea un interfaz gráfica que contega la siguiente funcionalidad:

\begin{itemize}
\item Una barra de menú, que tenga un menú denominado \emph{about}, con un item de menú denominado autor, que al activarlo muestre un \emph{joptionPane} que muestre el nombre del autor del proyecto.
\item Un \emph{textArea} que mostrará datos. Que cuando arranque debe aparecer la información del número de personas (en dígito) que contiene la tabla. Dicho valor debe proceder de una consulta de la base de datos. Los datos a mostrar no es un simple dígito, sino un mensaje tal y como \emph{Nº empleados en la empresa: 255}
\item Un botón denominado \emph{empleados jovenes} que debe mostrar el número de empleados menores de 25 años. Los datos deben venir de una cosulta y mostrarse en el \emph{textArea}
\item Un botón denominado \emph{datos ejemplos} que muestre en el \emph{textArea} el nombre y el email de las diez primeras personas. Los datos deben venir de una consulta.
\item Un botón denominado salir que cuando lo pulsemos cierre la aplicación pero previamente guarde en el fichero  \emph{menores.txt} la información que se muestra en el \emph{textarea} al iniciar la aplicación, mas la fecha actual. Ese fichero no debe sobreescribirse cada vez que se sale de la aplicación, se debe añadir dicha información.
\item Utliza un JOptionPane para separar la \emph{textarea} de los botones.
\end{itemize}

\item Utiliza el patrón \emph{singleton} para establecer la conexión.
\item Utiliza el patron DAO y DTO para el modelo.
\item Crea \emph{packages} para separar los componentes de la aplicación.
\end{itemize}
\section*{Cuestiones a tener en cuenta.}
\begin{itemize}
\item No hace falta cargar todos los datos de la base de datos en una lista. Debes hacer las consultas directamente a la base de datos.
\item Recomiendo el uso del patrón \emph{MVC}, en caso que no lo uses, comenta cada evento para facilitar su localización en el código.
\item En el patrón DAO, crea una interfaz donde definas los métodos que deben implementar la clase derivada, para su posterior uso en la interfaz gráfica.
\item Empieza creando la base de datos, comprobando el correcto funcionamiento de la misma.
\item Cuando implementes la clase \emph{DAO}, comprueba el correcto funcionamiento de las sentencias SQL.
\item Usa un \emph{PrintWriter} para crear el fichero de salida.
\end{itemize}
\section*{Para los que trabajen en Windows}
No usar variables ni comentarios con caracteres que no sean \emph{ASCII}, es decir que sobre todo no utilices acentos ni eñes.

\section*{Criterios de evaluación}
Los criterios de evaluación son los indicados a continuación:\par 
\vspace*{0.5cm}
\begin{tabular}{|c|c|}
\hline
\textbf{CRITERIO EVALUACIÓN} & \textbf{PUNTUACIÓN} \\
\hline
Creación de la base de datos & 1 pto.\\
\hline
Creación del modelo & 1.5 ptos.\\
\hline
Creación de la interfaz gráfica & 1.5 ptos.\\
\hline
Evento del botón \emph{empleados jovenes }& 1.5 ptos.\\
\hline
Evento del botón \emph{datos ejemplo}& 1.5 ptos.\\
\hline
Evento del botón salir & 2 ptos.\\
\hline
Uso del patrón MVC  & 1 ptos\\
\hline
\end{tabular}
\section*{Subida de ficheros}
Subir el \emph{workspace} comprimido a la plataforma y entregar una copia al profesor.
\end{document}
