\documentclass[a4paper,spanish]{article}
\usepackage[spanish]{babel}
\usepackage[ansinew]{inputenc}
\usepackage[T1]{fontenc}
\usepackage{graphicx}
\usepackage{multicol}
\usepackage{longtable}
\usepackage{array}
\usepackage{multirow}

\renewcommand{\tablename}{Tabla}
\author{Manuel Molino Milla \and Luis Molina Garz�n}
\title{\textbf{Programaci�n}
\\COLECCIONES}
\date{\today}

\begin{document}

\end{document}

\maketitle

\subsection*{Ejercicio 1}
Escribe un programa que implemente la clase Listas con las siguientes caracter�stacas:
\begin{itemize}
\item Tiene como atributos tres ArrayList, para guardar los nombre de mujeres, hombres y el total de ellos.
\item Un constructor que inicialice dichas lista a lista vacias.
\item M�todos que a�adan los nombres de hombres y mujeres desde un fichero de texto.
\item Un m�todo que cree la lista de todos los nombres fundiendo ambas listas.
\item M�todos que nos digan, dado un nombre si es mujer u hombre.
\item Todo aquellos que crees necesario.
\end{itemize}
Una vez implimentada la clase, comprueba su funcionamiento, pasando un par�metro que sea un nombre y nos devuelva si es v�lido ese nombre o no, as� como si es nombre de  mujer u hombre.
\subsection*{Ejercicio 2}
Escribe un programa que cree un fichero llamado \emph{ejercicio2.dat} si no existe, y que a�ada 100 n�meros enteros de forma aleatoria, usando clases I/O relacionadas con archivos binarios.\\
Posteriormente crea otro programa que lea dichos 100 n�meros y que muestre el valor de su suma.

\subsection*{Ejercicio 3}
Escribe un programa que lea lineas desde un fichero de texto y posteriormente las escriba dentro de un fichero binario denominado ejercicio3.dat\\
Posteriormente escribe otro programa que lea el fichero \emph{ejercicio3.dat}, y que nos diga el n�mero de bytes del fichero. Comprueba si coincide con los datos que tiene el sistema operativo.

\subsection*{Ejercicio 4}
Escibe una clase denominada Persona que tenga las siguientes caracter�sticas:
\begin{itemize}
\item Atributos: nombre, apellidos, edad y direcci�n.
\item Constructor que inicialice dichos atributos.
\item Los correspondientes \emph{getters y setters}
\item La clase debe ser serializable.
\end{itemize}
Crea un programa que cree cinco objetos del tipo Persona y que los guarde en un fichero binario denominado ejercicio4.dat.\\
Posteriormente crea otro fichero que lea dichos objetos y muestre por pantalla cada uno de los objetos.

\subsection*{Ejercicio 5}
Realiza un programa que divida un fichero en distintos trozos mas peque�os. La sint�xis de ejecuci�n ser�a:
\begin{quote}
\emph{java Leer ficheroEntrada numeroTrozos}
\end{quote}
Donde \emph{Leer} es es nombre del programa ejecutable en java, \emph{ficheroEntrada} es el fichero a dividir en trozos y \emph{numeroTrozos} es un numero que indica en el numero de partes en que se va a dividir el fichero.
\end{document}
