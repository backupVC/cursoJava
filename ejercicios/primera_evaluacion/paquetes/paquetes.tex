\documentclass[a4paper,spanish]{article}
\usepackage[spanish]{babel}
\usepackage[utf8x]{inputenc}
%\usepackage[T1]{fontenc}
\usepackage{graphicx}
\usepackage{multicol}
\usepackage{longtable}
\usepackage{array}
\usepackage{multirow}
\usepackage{hyperref}

\renewcommand{\tablename}{Tabla}
\author{Manuel Molino Milla \and Luis Molina Garzón}
\title{\textbf{Programación}
\\Packages}
\date{\today}

\begin{document}
\maketitle
\tableofcontents
\setlongtables


\section*{Fechas en Java}
\subsubsection*{Ejercicio 1}
Realiza un programa que lea la fecha actual utilizando las clases \emph{LocalDate} y \emph{LocalDateTime}, usando el método \emph{now}. Imprime el valor de la fecha anteriormente leída.
\\
Posteriormente formatea dicha fecha, para que tenga un formato en español.\\
Utiliza como referencia los ejemplos que aparecen en esta página:\\
\href{https://www.solvetic.com/tutoriales/article/1345-trabajando-con-fechas-en-java-8/}{Ejemplo práctico}

\subsection*{Creación de packages}
Crea un package denominido \emph{com.iesvirgendelcarmen.matemáticas} que tenga las dos siguientes clases:
\begin{itemize}
\item \emph{Geometria}, que tenga métodos estáticos para calcular el área de un triangulo, cuadrado, pentágono y hexágono.
\item \emph{Estadisticas}, que contenga métodos de instancia para calcular la media, la moda, la varianza y la desviación tipica.
\item En otro package denominado \emph{com.iesvirgendelcarmen.test} crea una clase TestMatemáticas donde pruebes el funcionamiento de las bibiliotecas anteriores.
\item En otro proyexto independiente crea un nuevo package denominado test e incluye una clase denominada \emph{TestMatemáticas} donde pruebes el funcionamiento de las bibiliotecas anteriores.

\end{itemize}

\subsection*{Creación de jar ejecutable}
Crea un jar ejecutable de los dos ejercicios anteriores. Para esto consulta información de como realizarlo en Eclipse.

\section*{Ejercicio bibliotecas externas}
\subsubsection*{Ejercicio 1}
Crea un package denominado \emph{com.iesvirgendelcarmen.ficheros} y un directorio en la ráiz de tu proyecto denominado \emph{pdf}, incorpora la librería \emph{itext} a tu proyecto y crea un documento de tipo pdf en la carpeta anterior y prueba el funcionamiento de dicha librería.\\
Para encotrar el fichero \emph{itext-X.Y.Z.zip}  visita la siguiente URL \url{http://developers.itextpdf.com/developers-home} donde también encontrarás la documentación pertinente.

\end{document}
