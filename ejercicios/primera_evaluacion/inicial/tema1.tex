\documentclass[a4paper,spanish]{article}
\usepackage[utf8x]{inputenc}
\usepackage[spanish]{babel}
\usepackage{graphicx}
\usepackage{multicol}
\usepackage{longtable}
\usepackage{array}
\usepackage{multirow}
\usepackage{hyperref}

\renewcommand{\tablename}{Tabla}
\author{Manuel Molino Milla \and Luis Molina Garzón}
\title{\textbf{Programación}
\\Introducción a la programación}
\date{\today}

\begin{document}
\maketitle \tableofcontents \setlongtables

\section{Ejercicios de repaso}
\subsubsection{Ejercicio 1}
Diferencia entre software y hardware. Pon ejemplos.
\subsubsection{Ejercicio 2}
Lista los componentes mas importantes de un computador.
\subsubsection{Ejercicio 3}
Define lenguaje máquina, lenguaje ensamblador y lenguaje de alto nivel.
\subsubsection{Ejercicio 4}
¿Qué es el código fuente de un programa? ¿Qué significa compilar un programa?
\subsubsection{Ejercicio 5}
¿Que es un sistema operativo?
\subsubsection{Ejercicio 6}
¿Cual es la diferencia entre un lenguaje compilado y un lenguaje interpretado.

\section{Informacion del equipo}
Busca información acerca de las características del equipo de trabajo: tipo de placa base, procesador, memoria, capacidad de disco, \dots
\section{Lenguajes de programacion}
Apoyandote en TIOBE Programming Community analiza los lenguajes
de programación mas populares. ¿Cómo se determina esta escala?
\section{Ejercicios sobre editores}
\subsection{Windows}
\begin{itemize}
\item Instala el editor \emph{Notepad++} en Windows y escribe el programa HolaMundo en java. Realiza lo mismo con el programa \emph{geany}
\end{itemize}
\subsection{Ubuntu}
Igual que para el sistema Windows, escribe el programa anterior en los siguientes editores:
\begin{itemize}
\item vim
\item geany
\item gedit
\end{itemize}

\newpage

\section{Lenguajes de programacion}
\subsection{Programando con editor propio}
Usando cualquiera de los editores anteriores, compila/ejecuta  o ejecuta el programa HolaMundo en los siguientes lenguajes de programacion:
\begin{itemize}
\item Python.
\item Pascal.
\item Perl.
\item C.
\end{itemize}

\subsection{Programando con editor online}
Usando un editor online, como puede ser \emph{\href{http://ideone.com/}{Ejecución código online}} que nos permite ejecutar programa, compila/ejecuta  o ejecuta el programa HolaMundo en los siguientes lenguajes de programacion:
\begin{itemize}
\item javascript
\item php
\item go
\item lua
\item kotlin
\item haskell
\end{itemize}
Indica qué tipo de lenguaje es, compilado o interpretado.

\end{document}
