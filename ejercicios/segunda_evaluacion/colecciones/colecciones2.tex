\documentclass[4paper]{article}
\usepackage[spanish]{babel}
%\usepackage[ansinew]{inputenc}
\usepackage[utf8x]{inputenc}
%\usepackage[utf-8]{inputenc}
%\usepackage[T1]{fontenc}
\usepackage{graphicx}
\usepackage{multicol}
%\usepackage{longtable}
%\usepackage{array}
%\usepackage{multirow}
%\usepackage[latin1]{inputenc}
%\inputencoding{latin1}
\usepackage{hyperref}
\renewcommand{\tablename}{Tabla}
\author{Manuel Molino Milla \and Luis Molina Garzón}
\title{\textbf{Programación}
\\IO}
\date{\today}

\begin{document}
\maketitle
%\tableofcontents
%\setlongtable 


\section*{Ejercicio 1}
El ejercicio a realizar consiste en la gestión de una base de datos sobre coches. Para obtener los datos, usando la URL \href{https://www.mockaroo.com/}{de mockaroo} genera un fichero de \emph{array de json} con los siguientes campos:
\begin{enumerate}
\item modelo del coche
\item fabricante del coche.
\item matrícula del coche
\end{enumerate}
Los campos \emph{model}o y \emph{fabricante} lo ofrece por defecto \emph{mockaroo} y en el caso de la matrícula usa una expresión regular con el formato cuatro números y tres letras.\\
Crea una clase base o modelo denominado \emph{Coche} con los siguientes atributos:
\begin{itemize}
\item Modelo.
\item Fabricante.
\item Matrícula.
\end{itemize}
Crea una clase auxiliar con los siguientes métodos:
\begin{itemize}
\item Un método estático que lea el fichero del \emph{array de json}, usando una librería externa \emph{gson} situandp a ésta en el proyecto en un directorio denominado \emph{lib} y obtén dicha librería usando el repositorio de \emph{maven}. Este método debe devolver un \emph{Map} con clave la matrícula y un \emph{array de String}, siendo el primer elemento del \emph{array} el modelo y el segundo el fabricante.
\end{itemize}
Una clase que gestione la conexión a la base de datos de acuerdo al patrón \emph{singleton}. Usa un fichero de propiedades para la gestión de la base de datos.\\
En relación a la clase que gestiona el patrón \emph{DAO}, tendrá como atributo un \emph{map} con las características anteriormente mencionadas e implementará los siguientes métodos:
\begin{itemize}
\item Un método que cree la tabla para la insercción de datos.
\item Un método que obtenga los datos de la base de datos, devolverá un objeto \emph{map} con la estructura antes indicada.
\item Un método que vuelque los datos a la base de datos usando los valores del atributo \emph{map}
\item Un método que borre los datos de la base de datos.
\item Un método que añada al \emph{map} una nueva entrada.
\item Un método que devuelva un objeto \emph{Coche} del \emph{map} dada una matrícula del coche.
\item Un método que borre una entrada del \emph{map} dada una matrícula del coche.
\item Un método que actualice el modelo y/o fabricante dada la matrícula del coche.
\end{itemize}
Una clase \emph{Test} que realice las siguientes operaciones:
\begin{itemize}
\item Crea un objeto \emph{DAO}, para eso hay que inicializar el atributo \emph{map} de dicha clase, teniendo en cuenta:
\begin{enumerate}
\item Si existe la base de datos y la tabla tiene datos, lo obtendremos de la base de datos.
\item En caso contrario lo leeremos del fichero usando el método correspondiente, además de crear la tabla de la base de datos y volcar los datos a dicha tabla.
\end{enumerate}
\item Crearemos un menú con las opciones \emph{CRUD} indicadas en el objeto \emph{DAO}
\item Una opción será salir del programa, la salida del programa conlleva el borrado de los datos de la tabla, el volcado de los nuevos datos y el cierre de la conexión de la base de datos. (Se puede obviar esto si no ha habido cambios en los datos, es decir solo se han hecho consultas).
\end{itemize}
En el caso que necesites algún método, si está relacionado con el objeto \emph{DAO} lo creas en dicha clase, y sino en la clase \emph{Auxilidar}
\end{document}
