\documentclass[4paper]{article}
\usepackage[spanish]{babel}
%\usepackage[ansinew]{inputenc}
\usepackage[utf8x]{inputenc}
%\usepackage[T1]{fontenc}
\usepackage{graphicx}
\usepackage{multicol}

\renewcommand{\tablename}{Tabla}
\author{Manuel Molino Milla \and Luis Molina Garzón}
\title{\textbf{Programación}
\\POO}
\date{\today}

\begin{document}
\maketitle \tableofcontents %\setlongtables


\section{Diagramas UML}
\subsection{Ejercicio 0}
Crea el diagrama UML de las siguientes clases:
\begin{itemize}
\item Cilindro.
\item Cuenta bancaria.
\end{itemize}
Utiliza la aplicacion \emph{dia} para realizar estos diagramas.\\
Por otra parte utiza los atributos y metodos que creas necesarios.

\section{Creación de objetos}
\subsection{Ejercicio 1}
Crea un programa que describa la clase \emph{Coche} con los siguientes atributos:
\begin{enumerate}
\item Cantidad de combustible en el depósito.
\item Consumo del coche a los 100 km.
\end{enumerate}
Y los siguientes métodos:
\begin{enumerate}
\item Un método para añadir combustible al depósito.
\item Conocer la cantidad de combustible que tiene el depósito.
\item Número de kilométro que puede recorrer con el combustible que tiene en ese momento.
\end{enumerate}
Cuando se crea un coche por defecto debe venir con 5 litros de gasolina y un cosumo de 7 litros a los 100 km (no están trucados). Crea dos objetos diferentes de la clase Coche. Añade 5 litros de gasolina a cada coche. Comprueba el funcionamiento de la clase creando una clase denominada \emph{TestCoche}\\
NO uses constructores en este ejercicio.


\section{Clases}
\subsection{Ejercicio 2}
Crea una clase denominada Libro, que recoja los atributos y metodos que creas conveniente.\par 
Comprueba su funcionamiento con una clase denominada \emph{TestLibro}



\subsection{Ejercicio 3}
Crea una clase denominada Palabra que tenga como unica variable de instancia el contenido de la palabra y un constructor que asigne dicho contenido al atributo anterior. Comprueba su funcionamiento y posteriormente crea los siguientes metodos:
\begin{itemize}
\item Un metodo que devuelva la palabra en mayúscula.
\item Un metodo que devuelva la palabra en minúscula.
\item Un metodo que devuelva el numero de letras que tiene dicha palabra.
\item Un metodo que reemplace las letras de la palabra. \emph{Ejemplo palabra cocodrilo, parametros \emph{o u}, valor devuelto cucudrilu}
\item Un metodo que devuelva la primera letra de la palabra.
\item Un metodo que devuelva la ultima letra de la palabra en mayuscula.
\end{itemize} 
Comprueba el funcionamiento de dichos metodos en una clase denominada \emph{TestPalabra}

\subsection{Ejercicio 4. API Math}
La clase Math de Java se utiliza para el manejo de funciones matemáticas. Busca información en la página oficial de Oracle y completa la siguiente información:\par
\vspace{0,5cm}
\begin{tabular}{|c|c|c|c|c|}
\hline
\bf{nombre método} & \bf{valor retorno} & \bf{parámetros} & \bf{Breve descripción} & \bf{Ejemplo} \\
\hline
abs & &  &  &  \\
\hline
max &  &  & &  \\
\hline
min &  &  & &  \\
\hline
random &  &  & &  \\
\hline
round &  &  & &  \\
\hline
sqrt &  &  & &  \\
\hline
cbrt &  &  & &  \\
\hline
\end{tabular}
\subsection{Ejercicio 5}
Crea una clase denominada Matematicas que tenga como unica variable de instancia un numero de tipo \emph{double}. Utiliza \emph{getter} y \emph{setter} que accedan y asignen valor al numero respectivamente. Comprueba su funcionamiento y posteriormente crea los siguientes metodos:
\begin{itemize}
\item Un metodo que devuelva la raiz cuadrada de dicho número.
\item Un metodo que devuelva la raiz cúbica de dicho número en valor absoluto. 
\item Un metodo que redondee el número de tipo double a entero. En el caso que sea un número negativo debe devolver su valor positivo redondeado.
\item Un metodo que devuelva número aleatorios en el intervalo de 0 al numero de la clase redondeado. Ejemplo si el valor del numero es 2.3, su redondeo es 2, por tanto debe devolver aleatoriamente los números 0, 1 y 2.
\end{itemize} 
Comprueba el funcionamiento tanto con números positivos como negativos.

\section{Miscelanea}
\subsection{Ejercicio 6}
Busca informacion sobre NaN, isInfinite( ) y isNaN( ) en Java. Propon ejemplos de su uso.
\subsection{Ejercicio 7}
Queremos programar con el paradigma de POO una clase que resuelva ecuaciones de segundo grado. Usa los atributos y métodos que creas oportuno. Usa un constructor para crear objetos de esta clase.\par 
Crea una clase denominada \emph{TestEcuacionSegundoGrado} para comprobar su correcto funcionamiento. 
\subsection{Ejercicio 8} 
Igual que antes, queremos un programa que implementa la clase \emph{TrianguloRectangulo} usando los atributos que consideres oportuno y métodos para devolver el valor de la hipotenusa, el áera del mismo, así como el perímetro de dicho triangulo.\par 
Comprueba el funcionamiento con una clase denominada \emph{TestTrianguloRectangulo}.\par 
Utiliza la \emph{API javax.swing.JOptionPane} para solicitar los valores de los catetos de dicho triangulo y también para mostrar los datos de los métodos creados.

\end{document}
