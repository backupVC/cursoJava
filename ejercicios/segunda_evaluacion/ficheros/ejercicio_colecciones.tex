\documentclass[4paper]{article}
\usepackage[spanish]{babel}
%\usepackage[ansinew]{inputenc}
\usepackage[utf8x]{inputenc}
%\usepackage[utf-8]{inputenc}
%\usepackage[T1]{fontenc}
\usepackage{graphicx}
\usepackage{multicol}
%\usepackage{longtable}
%\usepackage{array}
%\usepackage{multirow}
%\usepackage[latin1]{inputenc}
%\inputencoding{latin1}

\renewcommand{\tablename}{Tabla}
\author{Manuel Molino Milla \and Luis Molina Garzón}
\title{\textbf{Programación}
\\COLECCIONES}
\date{\today}

\begin{document}


\maketitle

\subsection*{Ejercicio 1}
Escribe un programa que lea dos ficheros de texto que se suministran comprimidos (nombres\_hombre.zip y nombre\_mujer.zip). Uno con el nombres de hombre y otro con nombre de mujeres. La clase debe denominarse \emph{Listas} con las siguientes características:
\begin{itemize}
\item Tiene como atributos tres ArrayList distintos, para guardar los nombre de mujeres, hombres y el total de ellos.
\item Un constructor que inicialice dichas lista a lista vacias.
\item Métodos que añadan los nombres de hombres y mujeres desde un fichero de texto que se te proporciona.
\item Un método que cree la lista de todos los nombres fundiendo ambas listas.
\item Métodos que nos digan, dado un nombre si es mujer u hombre.
\item Todo aquellos métodos que crees necesarios.
\end{itemize}
Una vez implimentada la clase, comprueba su funcionamiento.
\subsection*{Ejercicio 2}
Escribe un programa que lea el fichero de texto \emph{entrada.txt} con la intencion de calcular el numero de palabras que tiene y la frecuencia de repeticion de las mismas. Ten en cuenta lo siguiente:
\begin{itemize}
\item Crea tres atributos privados que sean tres colecciones diferentes, un ArrayList, un HashSet y un HashMap.
\item El constructor debe inicializar las colecciones sin elemento alguno.
\item Un método 	que añada las palabras del texto al ArrayList.
\item Los \emph{getters y setter que creas correspondiente}
\item Un método que calcule la frecuencia de las palabras del texto, para esto usa el método \emph{Collections.frequency}.
\item Un método que imprima el valor de las colecciones usando un \emph{Iterator}
\item Otro método que los imprima en orden.
\item Otro método que nos de el tamaño de las colecciones.
\end{itemize}
Para leer el fichero de texto, no debes leer signos de puntuación, usa si quieres el método \emph{replaceAll} de la clase \emph{String} que reemplaza dichos signos por cadena vacía. O bien usa cualquier estrategia en la lectura del texto que elimine dichos signos.

\end{document}
