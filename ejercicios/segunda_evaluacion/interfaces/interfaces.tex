\documentclass[a4paper,spanish]{article}
\usepackage[spanish]{babel}
\usepackage[ansinew]{inputenc}
\usepackage[T1]{fontenc}
\usepackage{graphicx}
\usepackage{multicol}
\usepackage{longtable}
\usepackage{array}
\usepackage{multirow}

\renewcommand{\tablename}{Tabla}
\author{Manuel Molino Milla \and Luis Molina Garz�n}
\title{\textbf{Programaci�n}
\\INTERFACES}
\date{\today}

\begin{document}

\maketitle

\subsection*{Ejercicio 1}
Consultando el API de \emph{java.lang.CharSequence} indica qu� m�todos debe desarrollar una clase que implemente dicha interfaz.

\subsection*{Ejercicio 2}
�Qu� est� mal en la siguiente interfaz?
\begin{verbatim}
public interface SomethingIsWrong {
    void aMethod(int aValue){
        System.out.println("Hi Mom");
    }
}
\end{verbatim}
Implementa dicha interfaz para que est� correcta.

\subsection*{Ejercicio 3}
�Est� correcta la siguiente interfaz?
\begin{verbatim}
public interface Marker {
}
\end{verbatim}

\subsection*{Ejercicio 4}
Escribe una clase que implemente la interfaz \emph{java.lang.CharSequence} La  implementaci�n debe devolver el String en orden inverso. Escribe un test que compruebe el funcionamiento de la clase, donde se compruebe todos los m�todos especificados en la interfaz.

\subsection*{Ejercicio 5}
Sup�n que tienes que dise�ar un servidor de sincronizaci�n de tiempo, que peri�dicamente tienes que informar a los clientes del tiempo y hora. Escribe una interfaz en java que el servidor debe utilizar para crear un protocolo con los clientes.

\subsection*{Ejercicio 6}
Construir una clase \emph{ArrayReales} que declare un atributo de tipo \emph{double[]} y que implemente una interfaz llamada \emph{Estadisticas}. El contenido de esta interfaz es el siguiente:
\begin{verbatim}
public interface Estadisticas {
   double minimo();
   double maximo();
   double sumatorio();
}
\end{verbatim}

\subsection*{Ejercicio 7}
Construir una clase \emph{final Math3} que ampl�e las declaraciones de m�todos est�ticos de la clase \emph{Math} y que implemente una interfaz llamada \emph{Extremos} compilada con el siguiente c�digo fuente.
\begin{verbatim}
public interface Extremos{
  int min(int [] a);
  int max(int [] a);
  double min(double [] a);
  double max(double [] a);
}
\end{verbatim}




\end{document}
