\documentclass[addpoints]{exam}
\usepackage[utf8x]{inputenc}
\usepackage{graphicx}
\pagestyle{empty}
\pointname{ punto}
\begin{document}
\begin{center}
\begin{Huge}
EXAMEN PRIMERA EVALUACIÓN
\end{Huge}
\vspace{0.06in}

\begin{huge}
PROGRAMACIÓN 1º DAM
\end{huge}\\
\vspace{0.09in}

\begin{LARGE}
Examen tipo B
\end{LARGE}
\vspace{0.1in}

\end{center}
\begin{center}
\fbox{\fbox{\parbox{5.5in}{\centering
Lee el enunciado del examen detenidamente y realiza los programas que se solicita y recuerda leer el apartado de subida de examen para conocer los archivos a entregar}}}
\end{center}


\vspace{0.1in}
\section{ENUCIADO}
Queremos realizar un programa informático, usando el paradigma de \emph{orientación a objetos} que represente a una persona con los siguientes atributos:
\begin{itemize}
\item Nombre.
\item Edad.
\item DNI.
\item Sexo: hombre(H) o mujer(M).
\item Peso.
\item Altura.
\end{itemize}
Por defecto, el DNI será  \emph{11111111}, edad \emph{18 años}, nombre \emph{Juan García Román}, sexo \emph{H}, peso \emph{70 kg} y altura \emph{1,70 m}.\\ 
Realiza un programa denominado \emph{Persona.java} que contemple los siguientes requerimientos:
\section{CUESTIONES}
\begin{questions}
\question[\half] Los atributos antes indicados.
\question[1] Se implementará varios constructores:
\begin{itemize}
\item Un constructor por defecto.
\item Un constructor con el nombre, edad y sexo, el resto por defecto.
\item Un constructor con todos los atributos como parámetro.
\end{itemize}
\question[2]
Se implementará los siguientes métodos:
\begin{description}
\item[calcularIMC()] devolverá el \emph{IMC}: índice de masa corporal (peso en kg dividido por altura en metros al cuadradro). Este será de visibilidad privada.
\item[estaEnPesoIdeal()] nos devolverá si el individuo está en peso ideal. Usará el método anterior. Un individuo normal está en peso normal si el \emph{IMC} se encuetra entre 18,5-24,99.
\item[esMayorEdad()] nos devuelve si tienes 18 años o mas.
\item[setDNI()] que genera un DNI aleatorio que será un número comprendido entre 0 y 999999.
\end{description}
\question[\half]
Sobreescribe el método \emph{toString()} con los datos que crees convenientes.

\question[1\half]
Posteriormente crea una clase denominada \emph{TesPersona} que cree tres objetos llamando a cada uno de los constructores definidos. Usando el método \emph{toString} presenta por pantalla los datos de esos tres objetos.\\
\question[1] Posteriormente con el último objeto comprueba si esa persona está en peso ideal, si es mayor de edad y cambia su DNI a un DNI aleatorio. Vuelve a llamar al método \emph{toString()} para mostrar por pantalla el nuevo objeto.
\question[1\half] Usando el programa \emph{dia} implementa el \emph{diagrama UML} de la \emph{Persona}
\end{questions}


\section{DOCUMETOS A ENTREGAR}
\begin{itemize}
\item Los ficheros fuentes \emph{Persona.java} y \emph{TestPersona.java}
\item El diagrama UML en formato \emph{png}
\item Todo los ficheros se comprimen en un único fichero denominado nombreApellidos.tar.gz o nombreApellidos.zip y se sube a la plataforma.  
\end{itemize}
\end{document}