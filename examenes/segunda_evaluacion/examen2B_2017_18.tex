\documentclass[addpoints,12pt]{exam}
\usepackage[spanish]{babel}
%\usepackage[latin1]{inputenc}
\usepackage[utf8x]{inputenc}
\pagestyle{empty}
\begin{document}
\begin{center}
\fbox{\fbox{\parbox{5.5in}{\centering {\LARGE EXAMEN TIPO B}\\El ejercicio se realizará en eclipse. Usa un workspace distinto al que usas de forma habitual. Crea un proyecto de java nuevo denominado Examen y posteriormente un package denominado ejercicio, donde desarrollar los programas}}}
\end{center}
\vspace{0.1in}
%\makebox[\textwidth]{Nombre:\enspace\hrulefill}
\begin{questions}
\question Queremos hacer un programa para controlar una agenda de contactos, para esto debemos desarrollar las siguientes clases con las especificaciones que se solicitan.
\begin{description}
\item[clase Contacto] Qué tiene los siguientes atributos y métodos:
\begin{itemize}
\item Nombre y apellidos
\item Teléfono móvil
\item Teléfono fijo.
\item Ciudad de residencia.
\item Un constructor que inicialice dichos atributos.
\item Los correspondientes \emph{getters}
\item Sobreescribe el método \emph{toString()} para que cuando tratemos la referencia de estos objetos como \emph{String}, nos muestre los datos con el siguiente formato:\\
Nombre: (nombre y apellidos) - Teléfono móvil: (móvil) - Teléfono fijo: (fijo)
\end{itemize}
\item[clase Agenda] Qué tiene los siguientes atributos y métodos:
\begin{itemize}
\item Nombre de la agenda.
\item Coleccion dinamica de contactos.
\item Un constructor que inicialice atributo nombre.
\item Un método que nos sirva para añadir objetos \textit{Contacto} a la coleccion de contactos.
\item Un método que nos devuelva una colección de objetos \emph{Contacto} cuando se le pasa como argumento el nombre de un lugar de residencia.  Es valido pasar el nombre tanto en mayusculas como en minusculas.
\item Un método que nos diga el numero de contactos . Pasamos como argumento el nombre del lugar de residencia, es valido pasar el nombre tanto en mayusculas como en minusculas.
\item Un método que pasamos un numero de telefono y nos devuelva el objeto \emph{Contacto} al que pertenece dicho numero. El numero puede corresponder a fijo o movil.
\item Sobreescribe el método \emph{toString} a tu gusto.
\item Puedes añadir cualquier método que veas necesario, aunque no es necesario para el desarrollo del ejercicio. Explica el porqué de su uso.
\end{itemize}
\item[FijoException] Nos sirve para el caso que intentemos crear un objeto \emph{Contacto} cuando el telefono no empiece por el numero 9.
\item[MovilException] Nos sirve para el caso que intentemos crear un objeto \emph{Contacto} cuando el telefono movil no empiece por el numero 6 o 7.
\item[TestAgenda] Qué tiene que realizar lo siguiente:
\begin{itemize}
\item Contener el método main.
\item Crear un objeto Agenda, con el nombre \emph{TRABAJO}.
\item Mediante la clase \emph{Scanner} leer los datos para crear tantos objetos \emph{Contacto} como sea posible, usando los datos aportados al final del documento. Observa que hay valores de telefono que harán saltar las correspondientes excepciones.
\item Usando programacion defensiva mediante expresiones regulares, no permitas datos que no sean cadenas para el nombre de los contactos y el lugar de residencia y los telefonos no sean nueve números, no hay que controlar si empiezan por 6, 7 o 9, para eso estan las excepciones
\item Añadir los objetos \emph{Contacto} válidos al objeto \emph{Agenda}
\item Mostrar en consola la referencia del objeto \emph{Agenda} de acuerdo a la sobreescritura del método \emph{toString}
\item Realizar la llamada al método \emph{obtenerDatosDeAgenda} que se detalla abajo.
\item Crear un método aparte del main, denominado \emph{obtenerDatosDeAgenda}, que tenga como argumento un objeto \emph{Agenda} y el  nombre de un lugar de residencia y que muestre en consola las llamadas de los métodos de la clase \emph{Agenda} que  muestran los objetos \emph{Contactos} de ese lugar de residencia y el número de contactos que pertenecen a ese lugar de residencia, también usando la llamada del método. Usan \emph{printf} para formatear la salida.
\end{itemize}
\end{description}
Consideraciones a tener en cuenta a la hora de hacer el ejercicio:
Criterios de evaluación:
\begin{description}
\item[2 pto.] Por el control correcto de las excepciones.
\item[1 pto.] Por la clase \emph{Contacto}
\item[2.5 ptos.] Por la clase \emph{Agenda}
\item[2.5 ptos.] Por la clase \emph{TestAgenda}
\end{description}
\textbf{Formato del fichero de subida:} se subirá el \emph{workspace} del proyecto del examen.
\end{questions}
\begin{verbatim}
     Ciudad           T. fijo      T. móvil             Ciudad
     Pedro García     910123456    610123456             Jaén
     Luis Fernández   910123457    610123457             Jaén
     María Fernández  910123458    610123458             Jaén 
     Error 1          9101234578   610123457             Jaén 
     Error2           910123457    10123457              Madrid
\end{verbatim}
\end{document}
