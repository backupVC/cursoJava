\documentclass[addpoints,12pt]{exam}
\usepackage[spanish]{babel}
%\usepackage[latin1]{inputenc}
\usepackage[utf8x]{inputenc}
\pagestyle{empty}
\begin{document}
\begin{center}
\fbox{\fbox{\parbox{5.5in}{\centering {\LARGE EXAMEN TIPO B}\\El ejercicio se realizará en eclipse. Usa un workspace distinto al que usas de forma habitual. Crea un proyecto de java nuevo denominado Examen.}}}
\end{center}
\vspace{0.1in}
%\makebox[\textwidth]{Nombre:\enspace\hrulefill}
\begin{questions}
\question Queremos hacer un programa para registrar los datos de alumnos que cursan enseñanza de grado medio, para esto debemos implementar las siguientes clases:
\begin{description}
\item[clase Alumno] Qué tiene los siguientes atributos y métodos:
\begin{itemize}
\item Nombre
\item Apellidos
\item Edad
\item Un constructor que inicialice dichos atributos.
\item Los correspondientes \emph{getters}
\item Sobreescribe el método \emph{toString()} para que retorno un \emph{String} con el siguiente formato:\\
APELLIDOS: apellidos\_alumno ::: NOMBRE: nombre\_alumno ::: EDAD: edad\_alumno
\end{itemize}
\item[clase Profesor] Qué tiene los siguientes atributos y métodos:
\begin{itemize}
\item Nombre
\item Primer apellido.
\item Si es profesor es definitivo o sustituto.
\item Un constructor que inicialice dichos atributos.
\item Los correspondientes \emph{getters}
\item Sobreescribe el método \emph{toString()} para que retorno un \emph{String} con el siguiente formato:\\
APELLIDO: apellido\_profesor ::: NOMBRE: nombre\_profesor ::: DEFINITIVO (true o false)
\end{itemize}
\item[clase Módulo] Qué tiene los siguientes atributos y métodos:
\begin{itemize}
\item Nombre.
\item Horas de duración. 
\item Un array para dos profesores que imparten el módulo.
\item Una colección de alumnos (objetos)
\item Un constructor que inicialice con el nombre y horas de duración como parámetros.
\item Los correspondientes \emph{getters}
\item Un método que devuelva una colección de alumnos menores de edad.
\item \emph{setters},  uno para establecer a los alumnos matriculados y otro para establecer los profesores. Para el caso de profesores, dicho método debe añadir a dos profesores, por lo que los argumentos serán dos objetos de tipo \emph{Profesor}.
\item Sobreescribe el método \emph{toString()} para que retorno un \emph{String} con el siguiente formato:\\
\emph{ALUMNOS:\\
lista de alumnos \\
PROFESORES:\\
lista de profesores\\
NOMBRE DEL MÓDULO\\
nombre\_mdulo\\
HORAS
horas\_mdulo}
\end{itemize}
\item[Curso] Clase usada para comprobar el funcionamiento de las tres clases anteriores. 
Debes completar el código que se te suministra.
\end{description}

Consideraciones a tener en cuenta a la hora de hacer el ejercicio:
\begin{itemize}
\item Debes implementar las clases anteriores.
\item Completa la clase \emph{Curso.java} de acuerdo a las instrucciones que incorpora dicho fichero.
\item Deberas entregar el proyecto de la siguiente manera:
\begin{enumerate}
\item El directorio \textbf{src} del proyecto.
\item El proyecto completo de eclipse. 
\end{enumerate}
\end{itemize}
Criterios de evaluación:
\begin{description}
\item[1.5 pto.] Por el archivo Alumno.java funcionando correctamente.
\item[1.5 pto.] Por el archivo Profesor.java funcionando correctamente.
\item[3.5 pto.] Por el archivo Modulo.java funcionando correctamente.
\item[3.5 ptos.] Por Curso.java funcionando correctamente.
\end{description}
\textbf{Formato del fichero de subida:} se subirá un archivo comprimido de la siguiente forma \emph{nombreApellidos.tar.gz} o \emph{nombre.Apellidos.zip}. incluyendo tanto el proyecto de eclipse como la carpeta \emph{src}.
\end{questions}
\end{document}