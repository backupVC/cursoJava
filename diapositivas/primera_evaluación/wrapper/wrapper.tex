\documentclass{beamer} 
\usetheme{default} 
\setbeamercovered{transparent}

%\useoutertheme{umbcfootline} 
\setbeamertemplate{background canvas}[vertical shading][bottom=red!20,top=yellow!30] 


\usepackage[spanish]{babel}
%\usepackage[latin1]{inputenc}
\usepackage[utf8x]{inputenc}
\usepackage{hyperref}
\usepackage{color}

\title{WRAPPER}

\author{Manuel J. Molino Milla \and Luis Molina Garzón}

\date{\today} %

\institute{IES Virgen del Carmen \and Departamento de Informática}




%\beamerdefaultoverlayspecification{<+->}

\begin{document}


\begin{frame}
  \titlepage
\end{frame}

\begin{frame}
    \frametitle{Logo}
\begin{figure}
\includegraphics[scale=1]{imagenes/logo.jpeg} 
\caption{Logo Java}
\end{figure}
\end{frame}

\begin{frame}
  \frametitle{Contenido}
  \tableofcontents[pausesections]
\end{frame}



\section{Introduccion}

\begin{frame}
    \frametitle{Introduccion}

\begin{itemize}[<+->]
      \item A veces es muy conveniente poder tratar los datos primitivos (int, boolean, \dots.) como objetos
      \item En el caso que usemos colecciones de objetos, estas colecciones solo pueden albergar objetos. Por lo tanto no puede almacenar tipos primitivos.
	   \item Para esto java proporciona \emph{wrapper} (envoltorios) para solucionar este problema.      
      \item Las clases envoltorio existentes son:
      \begin{enumerate}
\item \alert{Byte} para \alert{byte}
\item \alert{Short} para \alert{short}
\item \alert{Integer} para \alert{int}
\item \alert{Long} para \alert{long}
\item \alert{Boolean} para \alert{boolean}
\item \alert{Float} para \alert{float}
\item \alert{Double} para \alert{double}
\item \alert{Character} para \alert{char}
\end{enumerate}
      \end{itemize}
\end{frame}


\section{Wrapper}
\begin{frame}
\frametitle{Métodos interesantes}
\begin{description}[<+->]
\item[static int parseInt(String s)] Convierte la cadena \emph{String s} en un número entero. Ejemplo de uso: \emph{int i = Integer.parseInt("1");}
\item[static double parseDouble(String s)] Convierte la cadena \emph{String s} en un número double. Ejemplo de uso: \emph{double i = Double.parseInt("1");}
\item[boolean isInfinite()] nos dice si el número es \emph{infinito}. Ejemplo: double d = (2.0/0); System.outlprintln(''2.0/0 es infinito? ''+d.isInfinite());
\item[boolean isNaN()] nos dice si el número es \emph{NaN}. Ejemplo: double d = (0.0/0); System.outlprintln(''0.0/0 es NaN? ''+d.isNaN());
\end{description}
\end{frame}
\section{Ejemplo}
\begin{frame}[fragile]
\frametitle{Ejemplo}
\begin{verbatim}

public class Test{ 

   public static void main(String args[]){
      int x =Integer.parseInt("9");
      double c = Double.parseDouble("5");

      System.out.println(x);
      System.out.println(c);
   }
}
\end{verbatim}
\end{frame}

\section{Documentación}
\begin{frame}[fragile]
Mejor consultar la documentacion de API de Oracle Byte, Short, Integer, \dots
\vspace{1cm}
\pause
\begin{center}

\begin{Huge}
\textcolor{cyan}{FIN}
\end{Huge}
\end{center}
\end{frame}


\end{document}

