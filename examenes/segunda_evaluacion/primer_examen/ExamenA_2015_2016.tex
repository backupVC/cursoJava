\documentclass[addpoints,12pt]{exam}
\usepackage[spanish]{babel}
%\usepackage[latin1]{inputenc}
\usepackage[utf8x]{inputenc}
\usepackage{graphicx}
\pagestyle{empty}
\begin{document}
\begin{center}
\fbox{\fbox{\parbox{5.5in}{\centering {\LARGE EXAMEN TIPO A}\\Sigue las intrucciones que marca el ejercicio, no solo para la realización del mismo sino también para la entrega.\\\emph{No se permite ningun documento de ayuda, excepto los aportados por el profesor}\\La realización del ejercicio será en el \emph{IDE} Eclipse, utilizando un workspace vacío creando un proyecto llamado \emph{Examen} y un \emph{package} denominado \emph{\textit{segundaevaluacion}}}}}
\end{center}
\vspace{0.1in}
%\makebox[\textwidth]{Nombre:\enspace\hrulefill}

\begin{questions}


\question Crea una clase denominada \emph{Telefono} que tiene los siguientes atributos:
\begin{itemize}
\item Numero de telefono
\item Titular
\item Si es fijo o móvil
\end{itemize}
Crea -si lo ves necesario- los correspondeintes getters, setter, constructor o constructores y sobreescribir el  método \emph{toString}\\
Crea una excepción denominada \emph{TelefonoException} que se lanzará cuando el número de teléfono no tenga 9 dígitos, aunque puedes contemplar -de forma voluntaria- la opción de +34 delante de estos 9 dígitos (+34953505050). También se lanzará la excepción en el caso que el número de un móvil no empiece por 6 o 7, o en el caso de un fijo que no comience por 9. \\ 
Posteriormente crea una clase denominada \emph{Agenda} que tenga los siguientes atributos:
\begin{itemize}
\item Categoria (ejmplo: amigos, familia o trabajo)
\item Una colección de telefonos.
\end{itemize}
Incorpora en esta clase métodos para añadir y eliminar teléfonos. Genera los getters y setters correspondientes, si son necesarios, constructor o constructores y sobreescribes si lo consideras útil el método \emph{toString} 
\vspace{0.3cm}
\\
Crea una clase denomninada \emph{TestAgenda} que cree tres agendas de teléfono y que corresponden a las categorías de amigos, familia y trabajo. Lee los datos del CSV que se incorpora (inclúyelo en una carpeta denominada \emph{datos} en la raíz de tu workspace), de manera que si leemos la categoría amigos la incorporamos a la agenda de categoría amigos y así respectivamente.
\vspace{0.3cm}
\\
Una vez leido los datos debes mostrar por consola las tres agendas.
\vspace{0.3cm}
\\
Añade un método a la clase \emph{Agenda}, que dado un número de teléfono nos diga el titular de la línea, así como otro que haga lo contrario, es decir dado el titular nos diga el teléfono o telefonos, pues puede darse el caso que un titular tenga varios números de teléfono sean móviles o fijos. Deja una prueba en el \emph{main} de la clase \emph{TestAgenda} el correcto funcionamiento de estos dos métodos.\\
\newpage
\textbf{Criterios de evaluación}
\begin{description}
\item[1 pto.] Estructura correcta de la clase \emph{Telefono}.
\item[2 pto.] Funcionamiento correcto de la excepción esto incluye la validación de los números de teléfono en la clase \emph{Telefono}.
\item[1.5 pto.] Estructura correcta del clase \emph{Agenda}.
\item[2 pto.] Lectura del fichero con el Scanner. En el caso que no seas capaz de hacerlo, realiza la lógica en el método \emph{main} de la clase \emph{TestAgenda}, lee los datos desde la entrada estándar y crea una única agenda, en este caso se valorará este apartado con un solo punto.
\item[2 pto.] Por la correcta implementación de los métodos que buscan titulares o telefonos en la clase \emph{Agenda}
\item[0.5 ptos.] Diagrama UML de las clases \emph{Telefono} y \emph{Agenda} 
\item[0.5 ptos.] Creación del jar ejecutable de \emph{TestAgenda}.
\item[0.5 ptos.] Creación de la documentación de la clase \emph{Telefono} 
\end{description}
\end{questions}
%\newpage
\textbf{Formato del fichero de subida:}\\
Se subirá un archivo comprimido de la siguiente forma nombreApellidos.tar.gz o nombre.Apellidos.zip. En el debes incluir:
\begin{itemize}
\item El \emph{workspace} donde has realizado el examen que debe incluir ademas cuatro carpetas, una \emph{doc} para la documentación, otra \emph{jar} para el fichero ejecutable, otra \emph{datos} donde se localice el fichero \emph{csv} para la lectura de datos. y otra \emph{uml} para incorporar los diagramas UML en formato \emph{png}
\end{itemize}
\textbf{Consejos}
\begin{itemize}
\item Para la lectura de datos debes usar el constructor de la clase \emph{Scanner} que tenga como atributo un objeto de tipo \emph{File}, dicho objeto se crea con la clase \emph{File} y el constructor usa como parámetro un \emph{String} que corresponde a la ruta del fichero \emph{(File inFile = new File("path o ruta del fichero")}.
\item El fichero \emph{csv} incorpora teléfonos no válidos, por lo que debes gestionar correctamente la excepción.
\end{itemize}

\end{document}
