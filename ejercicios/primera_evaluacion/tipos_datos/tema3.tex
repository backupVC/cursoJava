\documentclass[4paper]{article}
\usepackage[spanish]{babel}
%\usepackage[ansinew]{inputenc}
\usepackage[utf8x]{inputenc}
%\usepackage[T1]{fontenc}
\usepackage{graphicx}
\usepackage{multicol}
%\usepackage{longtable}
%\usepackage{array}
%\usepackage{multirow}
\usepackage{amsmath}
%\renewcommand{\tablename}{Tabla}
\author{Manuel Molino Milla \and Luis Molina Garzón}
\title{\textbf{Tipo de datos}
\\Introducción a Java}
\date{\today}

\begin{document}
\maketitle
\tableofcontents
% \setlongtables
\newpage
\section{Ejercicos generales}
\subsection{Ejercicio 0}
¿Qué tipo de datos usarías para almacenar los siguientes datos?
\begin{itemize}
\item La velocidad de la luz.
\item La nota del curso 3.5
\item La nota media del curso.
\item El número de equipos  en el aula.
\item $2^{65}$
\item 234 \$

\end{itemize}
\subsection{Ejercicio 1}
Traslada el siguiente algoritmo a código Java:
\begin{itemize}
\item Declara una variable de tipo \emph{float} denominada millasRecorridadas.
\item Posteriormente inicialízala a \emph{100}
\item Declara una constante de tipo \emph{float} denominada MILLAS\_POR\_KILOMETRO con el valor 1.609
\item Declara una variable de tipo \emph{double} denominada kilometrosRealizados y asigna el valor de multiplicar los dos valores anteriores.
\item Muestra por pantalla los valores anteriores.
\end{itemize}
Repite el ejercicio, pero ahora cambia el tipo de la variable \emph{millasRecorridadas} y la constante MILLAS\_POR\_KILOMETRO a tipo \emph{double}, y la la variable \emph{kilometrosRealizados} a tipo \emph{float}
\subsection{Ejercicio 2}
Asume que \emph{int a = 1} y \emph{double d = 1.0} y que cada expresion es independiente de las anteriores. ¿Cual es el resultado de las siguientes expresiones?\\
Realiza el cálculo sin llevar un programa en java. Posteriormente realiza los cálculos en \emph{jshell}
\begin{itemize}
\item a = 46 / 9;
\item a = 46 \% 9 + 4 * 4 - 2;
\item a = 45 + 43 \% 5 * (23 * 3 \% 2);
\item a \%= 3 / a + 3;
\item d= 4 + d * d + 4;
\item d += 1.5 * 3 + (++a);
\item d -= 1.5 * 3 + a++;
\end{itemize}
\subsection{Ejercicio 3}
¿Cuales es el resultado de las siguientes operaciones?
\begin{itemize}
\item 56 \% 6
\item 78 \% -4
\item -34 \% 5
\item -34 \% -5
\item 5 \% 1
\item 1 \% 5
\end{itemize}
Compruebalo posteriormente en \emph{jshell}
\subsection{Ejercicio 4}
¿Qué operación utilizarías para la siguiente proposición:
\begin{verse}
Si hoy es lunes, qué día de la semana será dentro de 100 días.
\end{verse}

\subsection{Ejercicio 5}
Identifica y corrige los errores en el siguiente codigo:
\begin{verbatim}
public class Test {
   public void main(string[] args) {
   int i;
   int k = 100.0;
   int j = i + 1;

   System.out.println("j es " + j + " y k es " + k);
   }
}
\end{verbatim}



\subsection{Ejercicio 6}
Muestra la salida del siguiente codigo:
\begin{verbatim}
float f = 12.5F;
int i = (int)f;
System.out.println("f es " + f);
System.out.println("i es " + i);
\end{verbatim}
Compruebalo posteriormente en \emph{jshell}


\subsection{Ejercicio 7}
Muestra la salida del siguiente codigo:
\begin{verbatim}
System.out.println("1"+1);
System.out.println('1'+1);
System.out.println("1"+1+1);
System.out.println("1"+(1+1));
System.out.println('1'+1+1);
\end{verbatim}
Compruebalo posteriormente en \emph{jshell}

\subsection{Ejercicio 8}
¿Cuales de los siguientes identificadores es una constante, un metodo, una variable o una clase?
\begin{quote}
MAX\_VOR, Test, valor, leerEntero
\end{quote}


\section{Ejercicio de codigo}
\subsection{Ejercicio 1}
Escribe un programa que compute la siguiente expresion matematica:\\
\begin{center}
{\huge $\frac{9.5 * 4.5 - 2.5 * 3}{ 45.5 - 3.5}$}
\end{center}
\newpage
\subsection{Ejercicio2}  
Dado el siguiente código:
\begin{verbatim}
 /**
 * Esta clase define objetos que contienen 
 * un atributo denominado número
 * y posee un método para devolver
 * el valor doble
 * @author: Manuel Molino
 * @version: 1.0
 */
public class Numero{
   private int numero; //propiedad de los objetos.
   /**
   * Constructor: es un método que tiene
   * el mismo nombre que la clase y se
   * usa para crear objetos
   * @param n parámetro para inicializar el objeto
   * se usa para inicializar los objetos
   * de tipo número
   */
   public Numero(int n){
        this.numero=n;
   }
    /**
    * método que devuelve el valor doble
    * @return el valor doble del atributo
    * numero de dicho objeto.
    */
   public int cacularDoble(){
     return this.numero*2;
   }
}
//clase para comprobar el funcionamiento de la 
//clase anterior
class TestNumeros {
   public static void main(String[] arg){
       //Creo uno objeto de tipo Número
       Numero n1 = new Numero(4);
       //Imprimo su valor doble
       System.out.println("Valor doble: "+n1.calcularDoble());
   }
}
\end{verbatim}
\newpage
\begin{itemize}
\item ¿Cómo debes llamar al fichero que guarda dicho código?
\item ¿Cómo se compila? 
\item ¿Cuántos fichero de tipo \emph{class} genera?
\item ¿Cómo se ejecuta?
\item Genera la documentación con \emph{javadoc} y que quede ubicada en un directorio denominado \emph{doc}
\end{itemize}

\subsection{Ejercicio 2}
Modifica el fichero anterior y crea dos nuevos métodos que se llamen \emph{calcularTriple()} y \emph{calcularMitad()} de manera que se pueda mostrar su valor triple y la mitad de su valor. Añade los comentarios necesarios. Y vuelve a generar la documentación.

\subsection{Ejercicio 4}
Crea código como en el \emph{Ejercicio1} que nos sirva para convertir de euros a dólares y viceversa. Denomina la clase \emph{Moneda}

\subsection{Ejercicio 6}
Crea un nuevo programa, denominado Unidades.java que intercambie unidades del sistema metrico decimal y el el sistema anglosajon. Las unidades a intercambiar son:
\begin{itemize}
\item De libras a kilogramos.
\item De pies a metros.
\item De yardas a metros.
\item De millas a metros.
\item De acres a hectareas.
\item De onzas a mililitros.
\item De galones a litros.
\end{itemize}


\end{document}
