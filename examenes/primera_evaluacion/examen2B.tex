\documentclass[addpoints,12pt]{exam}
\usepackage[spanish]{babel}
%\usepackage[latin1]{inputenc}
\usepackage[utf8x]{inputenc}
\pagestyle{empty}
\begin{document}
\begin{center}
\fbox{\fbox{\parbox{5.5in}{\centering {\LARGE EXAMEN TIPO B}\\Sigue las intrucciones que marca el ejercicio, no solo para la realización del mismo sino también para la entrega.\\\emph{No se permite ningun documento de ayuda, excepto los aportados por el profesor}\\Se permite la realización del ejercicio en Eclipse o Netbeans}}}
\end{center}
\vspace{0.1in}
%\makebox[\textwidth]{Nombre:\enspace\hrulefill}
\begin{questions}
\question Queremos hacer un programa para registrar los datos de alumnos que cursan enseñanza de grado medio, para esto debemos implementar las siguientes clases:
\begin{description}
\item[clase Alumno] Qué tiene los siguientes atributos y métodos:
\begin{itemize}
\item Nombre
\item Apellidos
\item Edad
\item Un constructor que inicialice dichos atributos.
\item Los correspondientes \emph{getters y setters}
\item Y el metodo \emph{toString()} que imprima el objeto en el siguiente formato: Nombre Apellidos, ejemplo \emph{Federico Garcíez Jiménez}
\end{itemize}
\item[clase Profesor] Qué tiene los siguientes atributos y métodos:
\begin{itemize}
\item Nombre
\item Primer apellido
\item Edad
\item Un constructor que inicialice dichos atributos.
\item Los correspondientes \emph{getters y setters}
\item Y el metodo \emph{toString()} que imprima el objeto en el siguiente formato: Nombre Apellidos, ejemplo \emph{Federico Jiménez}
\end{itemize}
\item[clase Módulo] Qué tiene los siguientes atributos y métodos:
\begin{itemize}
\item Nombre.
\item Horas de duración. 
\item Profesor que imparte dicho módulo. (Sólo un profesor).
\item Un arrayList de alumnos (objetos)
\item Un constructor que inicialice los atributos anteriores y el arrayList vacio.
\item Los correspondientes \emph{getters y setters}
\item Un metodo que añada alumnos al atributo arrayList de objetos Alumno.
\item Un metodo que indique si un alumno esta matriculado o no en un modulo.
\item Y el metodo \emph{toString()} que imprima el objeto en el siguiente formato: Nombre del módulo y profesor que imparte, y alumnos. Ejemplo \emph{MODULO: Programacion, PROFESOR: Manuel Molino, ALUMNOS: [Juan Garcia Espeluy, Pedro Medina Montano, Juan Manuel Sanchez, Yasmina Haro Jazzel, Elena Peinado Contreras]}
\end{itemize}
\item[Curso] Qué tiene los siguientes atributos y métodos:
\begin{itemize}
\item Nombre.
\item Array de modulos.
\item Un constructor que inicialice los atributos de nombre y la array inicializado con un tamaño de tres.
\item Un método que añada modulos al curso.
\item Un método que nos devuelva el nombre del curso.
\item Y un método \emph{toString()} que imprima el objeto de la siguiente forma: Modulo, profesor del modulo y numero de alumnos en dicho modulo. Ejemplo MODULO: Programacion, PROFESOR: Manuel Molino, número de alumnos: 10. MODULO: Base de Datos: Joaquin Mendez, numero de alumnos: 7. MODULO: Entornos de programacion, PROFESOR: Fernando Cagancho, numero de alumnos 12. 
\end{itemize}
\item[TestCurso] Qué tenga en cuenta lo siguiente:
\begin{itemize}
\item El método main para iniciar el programa.
\item Crea seis alumnos.
\item Crea tres profesores.
\item Crea tres módulos.
\item Crea un curso.
\item De acuerdo al curso creado indica, de acuerdo a POO:
\begin{itemize}
\item Los datos del curso de acuerdo a su método toString().
\item Los alumnos de acuerdo a su método toStrig() que estén matriculado en los tres módulos a la vez.
\item El número de alumnos que estén matriculados en un solo módulo.
\end{itemize}
\end{itemize}
\end{description}
Consideraciones a tener en cuenta a la hora de hacer el ejercicio:
\begin{itemize}
\item Todas las clases deben pertenecer a un paquete denominado \emph{formacion}
\item Crea las clases de forma secuencial, primero la clase Alumno, luego la clase Profesor, luego la clase Modulo, luego la clase Curso y finalmente la clase TestCurso.
\item Cuando creas una clase, bien Alumno, Profesor, Modulo o Curso, incluye un método main para comprobar el correcto funcionamiento de la clase. Genera para esta clase un fichero \emph{jar ejecutable} (que posteriormente debes incluir en el examen) y mas tarde comenta dicho método main para que no interfiera en las posteriores clases. Guarda los ficheros jar que luego se te van a solicitar.
\end{itemize}
Criterios de evaluación:
\begin{description}
\item[2 ptos] Por los ficheros jar, son cinco correpondiente a las cinco clases. Cada uno de ellos correctamente funcionando son 0.4 ptos. Recuerda una vez que hagas, por ejemplo el de la clase Alumno o Profesor, comenta el método main para que no interfiera en el siguiente jar, el de la clase Modulo, y así sucesivamente.
\item[1 pto.] Por un fichero build.xml para la clase TestCurso.
\item[1 pto.] Si generas la documentación.
\item[1 pto.] Por la clase Alumno y Profesor, desarrollada correctamente.
\item[1 pto.] Por la clase Modulo, desarrollada correctamente.
\item[2 ptos] Por la clase Curso, desarrollada correctamente.
\item[2 ptos] Por la clase TestCurso, desarrollada correctamente.
\end{description}
Formato del fichero de subida: se subirá un archivo comprimido de la siguiente forma nombreApellidos.tar.gz o nombre.Apellidos.zip que incluya en la ráiz:
\begin{itemize}
\item build.xml
\item La carpeta src con los correspondientes fichero fuentes incluidos en el directorio del paquete (formacion).
\item La carpeta doc que incluye la correspondiente documentación de javadoc
\end{itemize}
\end{questions}
\end{document}
