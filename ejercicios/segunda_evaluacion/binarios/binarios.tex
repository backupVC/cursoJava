\documentclass[a4paper,spanish]{article}
\usepackage[spanish]{babel}
%\usepackage[ansinew]{inputenc}
%\usepackage[T1]{fontenc}
\usepackage[utf8x]{inputenc}
\usepackage{graphicx}
\usepackage{multicol}
\usepackage{longtable}
\usepackage{array}
\usepackage{multirow}

\renewcommand{\tablename}{Tabla}
\author{Manuel Molino Milla \and Luis Molina Garzón}
\title{\textbf{Programación}
\\FICHEROS BINARIOS}
\date{\today}

\begin{document}

\maketitle

\subsection*{Ejercicio 1}
Escribe un programa que cree un fichero llamado \emph{ejercicio1.txt}  y  añade 100 números enteros de forma aleatoria. Usa clases I/O relacionadas con archivos de texto (InputStreamReader)

\subsection*{Ejercicio 2}
Escribe un programa que cree un fichero llamado \emph{ejercicio2.dat} y  añade 100 números de tipo \emph{float} de forma aleatoria. Usa clases I/O relacionadas con archivos binarios (InputStream).\\
Posteriormente crea otro programa que lea dichos 100 números y que muestre el valor de su suma.

\subsection*{Ejercicio 3}
Escribe un programa que lea el fichero \emph{fundacionEImperio.txt} y nos diga el numero de lineas que contiene dicho fichero y cuantas palabras posee, en este caso consideramos una palabra cuando despues de esta aparece uno o mas espacios en blancos o un salto de linea.

\subsection*{Ejercicio 4}
Escibe una clase denominada Persona que tenga las siguientes características:
\begin{itemize}
\item Atributos: nombre, apellidos, edad y dirección.
\item Constructor que inicialice dichos atributos.
\item Los correspondientes \emph{getters y setters}
\item La clase debe ser serializable.
\end{itemize}
Crea un programa que cree cinco objetos del tipo Persona y que los guarde en un fichero binario denominado ejercicio4.dat.\\
Posteriormente crea otro programa que lea dichos objetos y muestre por pantalla cada uno de los objetos.

\subsection*{Ejercicio 5}
Realiza un programa que divida un fichero en distintos trozos mas pequeños. La sintáxis de ejecución sería:
\begin{quote}
\emph{java Leer ficheroEntrada numeroTrozos}
\end{quote}
Donde \emph{Leer} es es nombre del programa ejecutable en java, \emph{ficheroEntrada} es el fichero a dividir en trozos y \emph{numeroTrozos} es un numero que indica en el numero de partes en que se va a dividir el fichero. Usando \emph{Scanner} solicita el nombre de los ficheros de salida.
\end{document}
