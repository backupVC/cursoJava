\documentclass[addpoints,12pt]{exam}
\usepackage[spanish]{babel}
%\usepackage[latin1]{inputenc}
\usepackage[utf8x]{inputenc}
\usepackage{amssymb, amsmath, amsbsy} % simbolitos
\pagestyle{empty}
\begin{document}
\begin{center}
\fbox{\fbox{\parbox{5.5in}{\centering {\LARGE EXAMEN TIPO C}\\Sigue las intrucciones que marca cada ejercicio, no solo para la realización del mismo sino también para la entrega de los mismos.\\\emph{No se permite ningun documento de ayuda, excepto los aportados por el profesor}\\\emph{No se corregirá ningun ejercicio que tenga errores de compilación}}}}
\end{center}
\vspace{0.1in}
%\makebox[\textwidth]{Nombre:\enspace\hrulefill}
Crea una carpeta denominada \emph{examen}, creala tambien como repositorio de \emph{git} y crea un archivo \emph{REAME.md} que contenga como etiquetas tu nombre y apellidos. Crea un archivo \emph{.gitignore} para que \textit{git} ignore los archivos binarios de java y el directorio de documentación denominado \textit{doc}\par 
Crea un repositorio en tu \textit{GitHub} denominado \emph{examenPrimeraEvaluacion} y sincronízalo con el repositorio local

\begin{questions}
\question(4 ptos) Las ecuaciones físicas para la caída libre de un cuerpo son:\par
Cálculo de la velocidad con la que un cuerpo llega al suelo:
\begin{equation}
\sqrt{2 \cdot g \cdot h}
\end{equation}
Y el tiempo que tarda en llegar vale:
\begin{equation}
\sqrt{\frac{2 \cdot h}{g}}
\end{equation}
donde \emph{\textbf{g} es la aceleración de gravedad que vale 9.8} y \emph{\textbf{h} es la altura}\par 
Crea un programa en \emph{Java} denominado \emph{Fisica.java} que haga:
\begin{itemize}
\item Lea mediante la clase \emph{Scanner} el valor de la altura,
\item Comprueba que esa altura sea positiva y menor de 1000. En caso que no corresponda dentro de estos valores el programa terminara y mostrara un mensaje de error.
\item Crea una constante que defina la gravedad, denomínala \emph{ACELERACION\_GRAVEDAD}
\item Crea un método que devuelva el valor de la velocidad con la que un cuerpo llega al suelo.
\item Crea otro método que nos devuelva el tiempo que tarda en llegar al suelo.
\end{itemize}
Desglose de la puntuación
\begin{parts}
\part (0.5 ptos) Si usas correctamente el Scanner, en el caso que no sepas hacerlo introduce el valor de la altura como una variable local del método \emph{main}. En este caso no habrá puntuación en este apartado.
\part (1 pto) Por la comprobación y terminación del programa en el caso que el número solicitado no corresponda con las especificaciones.
\part (0.5 ptos) Por el método que calcula la velocidad de llegada al suelo 
\part (0.5 ptos) Por el método que calcula el tiempo que tarda en caer el cuerpo.
\part (0.5 ptos) Si usas \emph{printf} para mostrar las salidas en consola. Dicha salidas corresponde a las llamadas de los métodos.
\part (1 pto) Realiza la documentación de la clase que incluya las etiquetas \emph{author} y \emph{version}, mas la documentación de cada uno de los métodos, todo ello en una carpeta denomina \emph{doc}. No olvides hacer la documentación de la clase después de la sentencia \emph{import} de la clase \emph{Scanner}

\end{parts}
\newpage
\question (4.5 ptos) Realiza un programa -denominado \emph{Cadenas.java}- que lea mediante argumentos dos cadenas y realice lo siguiente:
\begin{itemize}
\item Comprobar que las cadenas tengan una longitud mínima de 4 caracteres. En caso contrario el programa acabara.
\item Un método que devuelva una cadena que proceda de la concatenación de las dos cadenas leídas.
\item Un método que nos devuelva la cadena de mayor longitud.
\item Un método que nos devuelva una de las cadenas en mayúscula.
\item Un método que nos digas cuantas veces la segunda cadena contiene el primer caracter de la primera cadena.
\item Un método que nos diga si la segunda cadena empieza por el último caracter de la primera cadena.
\end{itemize}
Desglose de la puntuación
\begin{parts}
\part (0.5 ptos) Si la lectura de las cadenas son correcta. En el caso que no sepas hacerlo introduce dichas cadenas como dos variables locales del método \emph{main}. En este caso no habrá puntuación en este apartado.
\part (1 pto) Por la comprobación y terminacion del programa
\part (0.5 ptos) Por el método que nos devuelve la cadena de mayor longitud.
\part (0.5 ptos) Por el método que nos devuelve la cadena en mayuscula.
\part (1 pto) Por el método que que nos digas cuantas veces la segunda cadena contiene el primer caracter de la primera cadena.
\part (1 pto) Por el método que nos diga si la segunda cadena empieza por el último caracter de la primera cadena.
\part (0.5 ptos) Si usas \emph{printf} en las llamadas a los métodos dentro del método \emph{main} de la clase principal

\end{parts}
\vspace{0.5 cm}
\question (1 ptos) GitHub
Crea los siguientes commmits:
\begin{itemize}
\item Un \emph{commit} denominado \emph{inicio del repositorio} que incluya el inicio del repositorio con los archivos \emph{README.md} y \emph{.gitignore}
\item Un \emph{commit} denominado \emph{ejercicio 1}, que incluya al fichero del código fuente del primer ejercicio.
\item Un \emph{commit} denominado \emph{documentación}, que la realizas cuando hayas realizado la documentación del primer ejercicio.
\item Un \emph{commit} denominado \emph{ejercicio 2}, que incluya al fichero del código fuente del segundo ejercicio.
\end{itemize}
Sincroniza el directorio local con el remoto.

\end{questions}
\newpage
\vspace{0,5cm}
Debes entregar los siguientes documentos:
\begin{itemize}
\item Fisica.java
\item Cadena.java
\item Fichero de texto cuyo contenido sea la \emph{URL} de tu repositorio del \emph{GitHub} del examen
\end{itemize}

Entrégalos en un documento comprimido con el formato:\emph{ apellidosNombre.tar.gz} o \emph{apellidosNombre.zip}. Tanto a la plataforma \emph{moodle} como al profesor.
\end{document}
