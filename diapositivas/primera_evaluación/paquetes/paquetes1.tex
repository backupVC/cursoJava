\documentclass{beamer} 
\usetheme{default}
\usecolortheme{albatross} 


%\useoutertheme{umbcfootline} 
%\setbeamertemplate{background canvas}[vertical shading][bottom=red!20,top=yellow!30] 


\usepackage[spanish]{babel}
%\usepackage[latin1]{inputenc}
\usepackage[utf8x]{inputenc}
\usepackage{multicol}


\title{Paquetes en Java}
\subtitle{Ocultando la implementación}
\author{Manuel J. Molino Milla \and Luis Molina Garzón}

\date{\today} %

\institute{IES Virgen del Carmen \and Departamento de Informática}




%\beamerdefaultoverlayspecification{<+->}

\begin{document}


\begin{frame}
  \titlepage
\end{frame}

\begin{frame}
    \frametitle{Logo}
\begin{figure}
\includegraphics[scale=1]{imagenes/logo.jpeg} 
\caption{Logo Java}
\end{figure}
\end{frame}

\begin{frame}
  \frametitle{Contenido}
  \tableofcontents[pausesections]
\end{frame}



\section{Introduccion}


\begin{frame}
    \frametitle{Introducción}

\begin{itemize}[<+-| alert@+>]
      \item Cuando se compila un fichero \emph{.java} se obtiene un fichero de salida que tiene exactamente el mismo nombre pero con extension \emph{.class}  
      \item Se pueden tener bastante ficheros \emph{.class} partiendo de un numero pequeño de fichero \emph{.java}
      \item Un programa en Java es un compendio de ficheros \emph{.java} que pueden empaquetarse y comprimise en un fichero \emph{.jar}      
      \item Para esto usaremos la herramienta \emph{jar} de Java.
      \item El intérprete de Java es el responsable de encontrar, cargar e interpretar estos ficheros.
      \item Si las clases están relacionadas se crean bibliotecas y a veces se empaquetan en un único fichero.
    \end{itemize}
    \pause
\end{frame}



\section{Package}



\begin{frame}[fragile]
    \frametitle{Bibliotecas}

\begin{itemize}[<+-| alert@+>]
      \item Una biblioteca tambien es un conjunto de estos fichero de clase.
      \item Generalmente cada fichero tiene una clase publica.
      \item Si todos los ficheros estan relacionados formamos un paquete \emph{package}
      \item Para eso ponemos como primera linea de cada fichero que forman la biblioteca:
      \item \emph{package nombre\_del\_paquete}.
      \item Aquellas clases que quieran utilizar esta biblioteca indicaran en la cabecera:
      \item \emph{import nombre\_del\_paquete}
    \end{itemize}
\end{frame}
\begin{frame}[fragile]
\frametitle{Ejemplos de package e import}
\begin{block}{Definicion del paquete}

    \begin{verbatim}
package mipaquete;
public class MiClase{ 
.....
\end{verbatim}
\end{block}

\pause
\begin{exampleblock}{Uso sin import}
\begin{verbatim}
mipaquete.MiClase m =  new mipaquete.MiClase();
.....
\end{verbatim}
\end{exampleblock}
\begin{alertblock}{Uso con import}

\pause
\begin{verbatim}
import mipaquete;
MiClase m =  new MiClase();
.....
\end{verbatim}
\end{alertblock}

\end{frame}

\begin{frame}
    \frametitle{Package}
    
\begin{itemize}[<+-| alert@+>]
\item Un package es una agrupación de clases afines.
\item Equivale al concepto de librería existente en otros lenguajes o sistemas.
\item Una clase puede definirse como perteneciente a un package y puede usar otras clases definidas en ese o en otros packages.
\item Los packages delimitan el espacio de nombres (space name).
\item El nombre de una clase debe ser único dentro del package donde se define.
\item Dos clases con el mismo nombre en dos packages distintos pueden coexistir e incluso pueden ser usadas en el mismo programa.
\end{itemize}
\pause
\end{frame}

\begin{frame}
    \frametitle{Package}
    
\begin{itemize}[<+-| alert@+>]
\item Los packages además también tienen un significado físico que sirve para almacenar los módulos ejecutables (ficheros con extensión .class) en el sistema de archivos del ordenador. 
\item Supongamos que definimos una clase de nombre \emph{miClase} que pertenece a un package de nombre \emph{misPackages.Geometria.Base}
\item Cuando la JVM vaya a cargar en memoria miClase  buscará el módulo ejecutable (de nombre miClase.class) en un directorio en la ruta de acceso \emph{misPackages/Geometria/Base}
\item Si una clase no pertenece a ningún package (no existe clausula package )  se asume que pertenece a un package por defecto sin nombre, y la JVM buscará el archivo .class en el directorio actual. 
\item Si una clase no se declara public sólo puede ser usada por clases que pertenezcan al mismo package.  
\end{itemize}
\pause
\end{frame}

\begin{frame}[fragile]
\frametitle{Paquetes}
Si compilamos el siguiente codigo con \alert{javac -d .}
\begin{verbatim}
package libreria.utilidades;
public class Clase{
    ......
}
\end{verbatim}
\pause 
Generamos automaticamente la siguiente estructura de directorios y archivos:
\pause
\begin{verbatim}
        directorio_raiz
             |
             |--- Clase.java
             |
             |--- libreria
             |        |
                      |---utilidades
                      |       |
                              |---Clase.class
\end{verbatim}

\end{frame}






\begin{frame}
    \frametitle{Creacion de paquetes unicos}
\begin{footnotesize}
\begin{itemize}[<+->]
\item La agrupacion de ficheros en paquete en un subdirectorio asegura nombre de paquetes unicos.
\item Tambien asegura una correcta localizacion de los mismos.
\item Para la creacion de paquetes unicos para su posterior uso se puede usar un dominio publico.
\item Ejemplo: package com.iesvirgendelcarmen.utilidades;
\end{itemize}
\end{footnotesize}
\pause
\end{frame}

\section{Modificadores de acceso en Java}
\begin{frame}
\frametitle{Modificadores usados}
Se colocan delante de cada declaración de cada miembro de la clase, bien sea un \emph{atributo} o bien un \emph{método}, controlando el acceso sólo para esa definción en particular.
\begin{itemize}[<+-|alert@+>]
\item Amistoso.
\item \emph{public}
\item \emph{private}
\item \emph{protected}
\end{itemize}
\pause
\end{frame}

\subsection{Amistoso}
\begin{frame}
\frametitle{Modificador amistoso}
\begin{itemize}[<+->]
\item Cuando no se define ningun modificador se dice que tiene acceso \emph{amistoso}.
\item Todas las clases del paquete actual tienen acceso al miembro amistoso. 
\item Pero las clases fuera del paquete no tienen acceso.
\item Este acceso da significado para agrupar juntas las clases de un paquete.
\end{itemize}
\pause
\end{frame}

\subsection{Public}
\begin{frame}
\frametitle{Modificador public}
\begin{itemize}[<+->]
\item Dicho atributo es accesible por todo el  mundo.
\item En especial al programador cliente que hace uso de las bibliotecas.
\end{itemize}
\pause
\end{frame}


\subsection{Private}
\begin{frame}
\frametitle{Modificador private}
\begin{itemize}[<+->]
\item Nadie puede acceder a ese miembro excepto a traves de los metodos de esa clase.
\item Otras clases del mismo paquete no pueden acceder a dichos miembros.
\item Se suelen usar para \emph{métodos ayudantes}
\end{itemize}
\pause
\end{frame}

\begin{frame}[fragile]
\frametitle{Acceso privado}
\begin{footnotesize}
\begin{verbatim}
        class Punto {
            private int x , y ;
            static private int numPuntos = 0;
            Punto ( int a , int b ) {
                x = a ; y = b;
                numPuntos ++ ;
            }
            int getX() {
                return x;
            }
            int getY() {
                return y;
            }
            static int cuantosPuntos() {
                return numPuntos;
            }
        }
\end{verbatim}  
\pause
\begin{center}
\alert{Obtenemos un ERROR si hacemos: \\
\pause
Punto p = new Punto(0,0);\\
p.x = 5;}
\end{center}
\end{footnotesize}
\end{frame}


\subsection{Protected}
\begin{frame}
\frametitle{Modificador protected}
\begin{itemize}[<+->]
\item Esta relacionado con el concepto de \emph{herencia}
\item Proporciona acceso público para las clases derivadas.
\item Y acceso privado (prohibido) para el resto de clases.
\end{itemize}
\pause
\end{frame}

\begin{frame}
\frametitle{Uso de tipo de accesos}
\begin{itemize}[<+->]
\item El modificador \emph{private} solo se aplica a miembros de clase (métodos y atributos) 
\item El modificador \emph{public} se aplica a clases y atributos de la clase.
\item El uso de modificadores \emph{públic y private} en variables locales originará error.
\item En la mayoría de los casos los constructores deberían ser \emph{public}
\item Podemos usar el modificador \emph{private} para constructores cuando queremos asegurarnos que no se puedan crear objetos de esa clase.
\item En el caso de la clase \emph{Math} donde sus miembros son \alert{static} podemos crear un constructor private:
\item private Math() \{ \}
\end{itemize}
\pause
\end{frame}

\begin{frame}
\frametitle{Buenas prácticas}
\begin{itemize}[<+->]
\item Se aconseja declarar todos los atributos como \emph{private}
\item Cuando necesitemos consultar su valor o modificarlo, utilicemos los métodos \emph{get} y \emph{set}.
\item En caso de que el tipo de valor devuelto por el método sea un boolean, se utilizará \emph{is} en vez de \emph{get}.
\end{itemize}
\pause
\end{frame}


\begin{frame}[fragile]
\frametitle{Ejemplo 1}
\begin{verbatim}
//Atributos private
private String nombre;
private boolean casado;
 
//Metodos get y set para estos atributos
public void setNombre(String nombre) {
    this.nombre = nombre;
}
public void setCasado(boolean casado) {
    this.casado = casado;
}
public String getNombre() {
    return nombre;
}
public boolean isCasado() {
    return casado;
}
\end{verbatim}
\end{frame}

\begin{frame}[fragile]
\frametitle{Ejemplo 2}
\begin{footnotesize}
\begin{verbatim}
public class Vehiculo {
    //Atributos con acceso private
    private String modelo;
    private int velocidad;
    private boolean arrancado;
    //El constructor siempre debe de ser public
    public Vehiculo(String modelo, int velocidad, boolean arrancado) {
        this.modelo = modelo;
        this.velocidad = velocidad;
        this.arrancado = arrancado;
    }  
    //Atributos getter con acceso public
    public String getModelo() {
        return modelo;
    } 
    public int getVelocidad() {
        return velocidad;
    } 
    public boolean isArrancado() {
        return arrancado;
    }
}
\end{verbatim}
\end{footnotesize}
\end{frame}

\section{Archivos jar}
\begin{frame}
\frametitle{Archivos jar}
\begin{itemize}[<+->]
\item Los ficheros Jar \emph{(Java ARchives)} permiten recopilar en un sólo fichero varios ficheros diferentes, almacenándolos en un formato comprimido para que ocupen menos espacio.
\item La particularidad de los ficheros .jar es que no necesitan ser descomprimidos para ser usados.
\item El intérprete de Java es capaz de ejecutar los archivos comprimidos en un archivo jar directamente. 
\item Ejemplo:
\item \alert{java -jar aplic.jar}
\end{itemize}
\pause
\end{frame}

\begin{frame}
\frametitle{Crear jar}
\begin{itemize}[<+->]
\item Si el fichero .jar contiene muchos ficheros .class, ¿cuál de todos se ejecuta?
\item Sabemos que debe ejecutarse el que contiene el método main pero, ¿cómo sabe el intérprete de Java que clase de todas las del fichero contiene este método?
\item La respuesta es que lo sabe porque se lo tenemos que decir durante la creación del fichero jar.  
\item Para crearlo debemos seguir los siguientes pasos:
\begin{enumerate}
\item Crear un directorio META-INF (las mayúsculas son necesarias) y dentro un fichero MANIFEST.MF
\item En el fichero \emph{MANIFEST.MF} ponemos: Main-Class: Principal
\item Creamos el fichero: \alert{\emph{jar cfm fich.jar META-INF/MANIFEST.MF *.class}}
\item Podemos saber el contenido del fichero: \alert{\emph{jar tf fich.jar}}
\item Lo lanzamos con: \alert{\emph{java -jar fich.jar }}
\end{enumerate}
\item De todas formmas los IDE acutales lo hacen automáticamente.
\end{itemize}
\pause
\end{frame}

\section{Miscelánea}
\begin{frame}
\frametitle{API java Fechas}
Las nuevas clases que se aportan en \emph{Java 8}:
\begin{description}[<+->]
\item[Instant] es, básicamente, un timestamp numérico. Es una clase útil para realizar logs y usar para frameworks de persistencia.
\item[LocalDate] sirve para almacenar una fecha sin hora. Por ejemplo, un cumpleaños como ''16-12-1979''
\item[LocalTime] sirve para almacenar una hora sin fecha. Por ejemplo, un horario de apertura a las ''9:30''.
\item[LocalDateTime] sirve para almacenar una fecha con hora. Por ejmplo, puede almacenar ''1979-12-16T12:30''
\item[ZonedDateTime] almacena hora y fecha con información de uso horario.
\end{description}
\end{frame}

\begin{frame}[fragile]
\frametitle{Ejemplo}
\begin{footnotesize}
\begin{verbatim}
import java.time.*;
public class Fechas{
   public static void main(String[] arg){
      //imprimir la fecha actual
      LocalDate date = LocalDate.now();
      System.out.println(date);
      System.out.printf("%s-%s-%s%n", date.getDayOfMonth(),
            date.getMonthValue(),  date.getYear());
      //vamos a sumar 5 horas
      LocalTime time = LocalTime.now();
      System.out.println(time);
      LocalTime newTime;
      newTime = time.plusHours(5);
      System.out.println(newTime);
      //imprimimos fecha y hora
      LocalDateTime dateTime = LocalDateTime.now();
      System.out.println(dateTime);
   }
}
\end{verbatim}
\end{footnotesize}
\end{frame}

\begin{frame}
\frametitle{Preguntas} 
\begin{figure}
\includegraphics[scale=0.9]{imagenes/dudas.png} 
\caption{Lenguaje máquina}
\end{figure} 
\end{frame}

\end{document}

