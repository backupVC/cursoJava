\documentclass[a4paper,spanish]{article}
\usepackage[spanish]{babel}
\usepackage[ansinew]{inputenc}
\usepackage[T1]{fontenc}
\usepackage{graphicx}
\usepackage{multicol}
\usepackage{longtable}
\usepackage{array}
\usepackage{multirow}

\renewcommand{\tablename}{Tabla}
\renewcommand{\S}{Base de datos y Java}
\author{Examen tercera evaluaci�n}
\title{\textbf{\S}}
\date{\today}

\begin{document}
\maketitle 
%\tableofcontents
%\newpage
\section*{Creacion de la BD}
El ejercicio consiste en la creaci�n de una base de datos llamada Mundo. La debes crear en \emph{mysql} con integridad referencial y con las siguientes tablas (crea un script para la creaci�n de dichas tablas para posterior entrega en el examen):
\begin{description}
\item[continentes] que contiene un \emph{id} como clave primaria con autoincremento y el nombre del continente.
\item[paises] que tiene los siguiente campos:
\begin{enumerate}
\item \emph{id} como clave primaria, de tipo \emph{VARCHAR}.
\item \emph{nombre} que corresponde al nombre del pais.
\item \emph{idioma} que corresponde al idioma hablado en dicho p�is.
\item \emph{idContinente} que corresponde a una clave foranea de la tabla continente.
\end{enumerate}
\end{description}
Vuelca los datos que se te proporcionan en el fichero datos.sql\par 
Crea una vista que dado el nombre de un idioma nos diga el nombre de los paises donde se habla y el nombre del continente al que pertenece (A��dela al script anterior de la creaci�n de la BD).\par 
Crea dos usuarios \emph{cliente y administrador} el primero s�lo con permisos de consulta y el otro con permisos de selecci�n, insercci�n, borrado y actualizaci�n. (A��de las sentencias al script anterior de la creaci�n de la BD)

\section*{Creacion de clases}
Crea las clases siguientes usando \emph{jdbc}. En cada clase se conecta y se desconecta de la BD. Debes usar uno de los dos usuarios para cada clase, seg�n veas conveniente:
\begin{itemize}
\item Una clase por cada tabla con sus datos correspondientes.
\item Una clase que facilite la consulta del numero total de paises que tiene un continente, valor que es solicitado en un scanner en la clase \emph{TestMundo.java}.
\item Una clase que imprima el nombre de todos los paises que empiezan por \emph{C}
\item Una clase que solicite por la entrada est�ndar (con scanner y en la clase \emph{TestMundo.java}) el nombre del \emph{idioma} nos devuelva el nombre del pa�s y continente. (Aprovecha la vista que has usado).
\item Una clase que a�ada un nuevo p�is. Los datos se recogen mediante un scanner en \emph{TestMundo.java}, para el caso del continente se debe introducir el nombre del continente, nunca el \emph{id} de dicho continente. 
\item Una clase que borre los paises \emph{Espa�a} y \emph{Portugal} y cree un nuevo pa�s denominado \emph{Iberia} con \emph{id} denominado \emph{IB}. Ten encuenta que las ejecuciones de las sentencia abarcan una transacci�n.
\item Crea una clase denominada TestMundo que contenga el metodo \emph{main} que pruebe todos los metodos anteriormente expuestos.
\end{itemize}

\section*{Optimizacion}
Escribe en una clase denominada Cuestion.java que imprima por pantalla como se puede optimizar esta BD.

\section*{Criterios de evaluacion}
Los criterios de evaluaci�n son los indicados a continuaci�n:\par 
\vspace*{0.5cm}
\begin{tabular}{|c|c|}
\hline
\textbf{CRITERIO EVALUACION} & \textbf{PUNTUACION} \\
\hline
Creacion de tablas e inserccion de datos & 2 ptos.\\
\hline
Vista & 0.5 ptos.\\
\hline
Usuarios & 0.5 ptos.\\
\hline
TestMundo.java y resto de java no asociados a la BD & 1 pto.\\
\hline
Clase para conocer el numero de paises  & 1 pto.\\
\hline 
Clase para conocer los paises que empiezan por C & 1 pto.\\
\hline
Clase para conocer el nombre del pais dado el idioma & 1 pto.\\
\hline
Clase para la inserccion de un pais & 1 pto\\
\hline
Clase conteniendo la transaccion & 1 pto.\\
\hline
Clase sobre optimizaci�n & 1 pto\\
\hline
\end{tabular}
\section*{Subida de ficheros}
Debes subir el proyecto de eclipse junto al script de la BD.
\section*{Ayuda}
Para la conexion de la BD en \emph{java} mediante \emph{jdbc} ten encuenta las siguientes sentencias:
\begin{quote}
Class.forName(''com.mysql.jdbc.Driver'');\\
Connection conexion = DriverManager.getConnection (''jdbc:mysql://localhost/nombreBD'',''usuario'', "clave'');\\
\end{quote}
Tambi�n puedes consultar las siguientes direcciones web:
\begin{quote}
http://http://www.mysqltutorial.org//\\
http://docs.oracle.com/javase/tutorial/jdbc/basics/index.html
\end{quote}
\end{document}
