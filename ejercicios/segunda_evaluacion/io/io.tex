\documentclass[a4paper,spanish]{article}
\usepackage[spanish]{babel}
\usepackage[ansinew]{inputenc}
\usepackage[T1]{fontenc}
\usepackage{graphicx}
\usepackage{multicol}
\usepackage{longtable}
\usepackage{array}
\usepackage{multirow}

\renewcommand{\tablename}{Tabla}
\author{Manuel Molino Milla \and Luis Molina Garz�n}
\title{\textbf{Programaci�n}
\\Entrada y Salida}
\date{\today}

\begin{document}
\maketitle 


\section*{Ejercicio 1}
Escribe una clase denominada PosiblePassword, que tenga como �nico atributo un String. Una contrase�a es v�lida de acuerdo a las siguientes reglas:
\begin{itemize}
\item Tenga al menos 8 caracteres.
\item Debe contener letras y n�meros obligatoriamente.
\item Debe contener al menso una letra en may�scula.
\item Como m�nimo dos digitos.
\end{itemize}
Crea una clase denominada TestPassword que lea por teclado contrase�as y nos diga si son v�lidas o no.
\section*{Ejercicio 2}
Un anagrama es una palabra o frase que resulta de la transposici�n de letras de otra palabra o frase. Por ejemplo:
\begin{quote}
AMOR - ROMA - OMAR - MORA - RAMO - ARMO - MARO\\
MONJA - JAM�N - MOJAN\\
L�MINA - ANIMAL\\
ESPONJA - JAPON�S
\end{quote}
Escribe una clase PosibleAnagrama, con los atributos que consideres oportunos y que compruebe si dos palabras son o no un anagrama.\par 
Escribe una clase denominada TestAnagrama que lea por pantalla dos palabras y nos diga si son o no anagramas.

\section*{Ejercicio 3}
Crea una clase denominada Parse.java que dado un fichero html \emph{descargado de Internet} nos de el n�mero de tags h1 que posee.

\section*{Ejercicio 4}
Crea un un programa que se ejecute de la siguiente forma:
\begin{quote}
java Cambiar viejoFichero nuevoFichero palabra
\end{quote}
Donde:
\begin{description}
\item[viejoFichero] es el fichero de entrada inicial.
\item[palabra] es la \emph{palabra} que debe convertirse en mayuscula.
\item[nuevoFichero] es el fichero de salida, donde se han reemplazado las palabras a mayusculas.
\end{description}
Tambien en \emph{nuevoFichero} debe aparecer al final el numero de palabras reemplazadas y el numero de palabras totales del fichero.

\section*{Ejercicio 5}
Crea un programa que lea por pantalla distintas letras, para averiguar una palabra previamente guardada en el programa. Ejemplo de ejecuci�n del programa:
\begin{verbatim}
******
m
m*****
q
m*****
a
ma****
o
ma*o*o
l
ma*olo
m
ma*olo
n
manolo
\end{verbatim}
\end{document}
