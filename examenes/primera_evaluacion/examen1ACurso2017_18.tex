\documentclass[addpoints,12pt]{exam}
\usepackage[spanish]{babel}
%\usepackage[latin1]{inputenc}
\usepackage[utf8x]{inputenc}
\pagestyle{empty}
\begin{document}
\begin{center}
\fbox{\fbox{\parbox{5.5in}{\centering {\LARGE EXAMEN TIPO A}\\Sigue las intrucciones que marca cada ejercicio, no solo para la realización del mismo sino también para la entrega de los mismos.\\\emph{No se permite ningun documento de ayuda, excepto los aportados por el profesor}\\\emph{No se corregirá ningun ejercicio que tenga errores de compilación}}}}
\end{center}
\vspace{0.1in}
%\makebox[\textwidth]{Nombre:\enspace\hrulefill}
Crea una carpeta denominada \emph{examen}, creala tambien como repositorio de \emph{git} y crea un archivo \emph{REAME.md} que contenga como etiquetas tu nombre y apellidos. Crea un archivo \emph{.gitignore} para que \textit{git} ignore los archivos binarios de java y el directorio de documentación denominado \textit{doc}\par 
Crea un repositorio en tu \textit{GitHub} denominado \emph{examenPrimeraEvaluacion} y sincronízalo con el repositorio local

\begin{questions}
\question(4.5 ptos) Realiza el siguiente programa de cálculo numérico -denominado \emph{Numero.java}- en el cual el programa reciba por argumentos un número entero. Dicho argumento debera pasarse a tipo \textit{int} usando la clase \textit{Integer}.
\begin{itemize}
\item Se debe comprobar que dicho número es mayor que cuatro y menor de 1000. En caso contrario el programa terminará indicando en pantalla que el número introducido no es válido.
\item Se comprobará si dicho número es par o impar, mostrando en consola dicha afirmación.
\item Se comprobará si es múltiplo de tres y cinco a la vez, e igual que antes mostraremos dicha afirmación.
\item Nos diga cuantos dígitos tiene, para esto puede usar métodos de la clase \emph{String} o bien usar cualquier algoritmo que se te ocurra. Tienes que mostrar dicho resultado en pantalla
\end{itemize}
Desglose de la puntuación
\begin{parts}
\part (1 pto) Si lees el número como argumento y lo conviertes a \emph{int}. En el caso que no sabes hacerlo, introducelo directamente en el programa y esta parte queda sin valorar. 
\part (1 pto) Por la comprobación y terminación del programa en el caso que el número solicitado no corresponda con las especificaciones.
\part (0.5 ptos) Por el métodos que nos dice si es par o impar. 
\part (0.5 ptos) Por el métodos que nos dice si es múltiplo de acuerdo a lo indicado anteriormente.
\part (1 pto) Por el método que nos diga el número de cifras del número.
\part (0.5 ptos) Si usas \emph{printf} para mostrar las salidas en consola.

\end{parts}
\newpage
\question (4.5 ptos) Realiza un programa -denominado \emph{Cadena.java}- que lea mediante la clase \emph{Scanner} una cadena en el método \emph{main} e implementa los siguientes métodos (los cuáles se llamarán en este método main):
\begin{itemize}
\item Un método que nos diga si dicha cadena empieza o acaba por vocal.
\item Un método que nos cuente el número de vocales no acentuadas que contenga dicha cadena.
\item Nos diga si contiene alguna consonante siguiente (m,n,p)
\item Un método que nos diga si la cadena empieza o no por vocal.
\end{itemize}
Desglose de la puntuación
\begin{parts}
\part (0.5 ptos) Si la lectura del scanner es correcta. Si no eres capaz de usar la clase \emph{Scanner}, introduce directamente el valor de la cadena en una variable de tipo \emph{String}, en este caso no habrá puntuación alguna.
\part (1 pto) Por el método que nos cuenta el número de vocales no acentuadas.
\part (0.5 ptos) Por cada uno de los otros métodos.
\part (0.5 ptos) Si usas \emph{printf} para mostrar las salidas en consola.
\part (1 pto) Realiza la documentación de la clase que incluya las etiquetas \emph{author} y \emph{version}, mas la documentación de cada uno de los métodos, todo ello en una carpeta denomina \emph{doc}. No olvides hacer la documentación de la clase después de la sentencia \emph{import} de la clase \emph{Scanner}
\end{parts}
\vspace{0.5 cm}
\question (1 ptos) GitHub
Crea los siguientes commmits:
\begin{itemize}
\item Un \emph{commit} denominado \emph{inicio del repositorio} que incluya el inicio del repositorio con los archivos \emph{README.md} y \emph{.gitignore}
\item Un \emph{commit} denominado \emph{ejercicio 1}, que incluya al fichero del código fuente del primer ejercicio.
\item Un \emph{commit} denominado \emph{ejercicio 2}, que incluya al fichero del código fuente del segundo ejercicio.
\item Un \emph{commit} denominado \emph{documentación}, que la realizas cuando hayas realizado la documentación del segundo ejercicio.
\end{itemize}
Sincroniza el directorio local con el remoto.

\end{questions}
\newpage
\vspace{0,5cm}
Debes entregar los siguientes documentos:
\begin{itemize}
\item Numero.java
\item Cadena.java
\item Fichero de texto cuyo contenido sea la \emph{URL} de tu repositorio del \emph{GitHub} del examen
\end{itemize}

Entrégalos en un documento comprimido con el formato:\emph{ apellidosNombre.tar.gz} o \emph{apellidosNombre.zip}. Tanto a la plataforma \emph{moodle} como al profesor.
\end{document}
