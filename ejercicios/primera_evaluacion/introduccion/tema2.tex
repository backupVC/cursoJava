\documentclass[4paper]{article}
\usepackage[spanish]{babel}
%\usepackage[ansinew]{inputenc}
\usepackage[utf8x]{inputenc}
%\usepackage[T1]{fontenc}
\usepackage{graphicx}
\usepackage{multicol}
%\usepackage{longtable}
%\usepackage{array}
%\usepackage{multirow}

%\renewcommand{\tablename}{Tabla}
\author{Manuel Molino Milla \and Luis Molina Garzón}
\title{\textbf{Programación}
\\Introducción a Java}
\date{\today}

\begin{document}
\maketitle
\tableofcontents
% \setlongtables
\section{Ejercicios de repaso}
\subsection{Ejercicio 1}
¿Qué significa que un lenguaje de programación sea \emph{case sensitive}.? ¿Se puede aplicar a Java?
\subsection{Ejercicio 2}
¿Qué se necesita en un computador para correr un programa de java?
\subsection{Ejercicio 3}
¿Cuáles son la entrada y salida en un compilador java?
\subsection{Ejercicio 4}
¿Qué es JVM?
\subsection{Ejercicio 5}
¿Qué son las palabras reservadas de un lenguaje de programación? Indica algunas que hayas visto en Java.
\subsection{Ejercicio 6}
¿Cuál es la extensión de un fichero fuente de Java? ¿Y del fichero que contiene los bytecodes?
\subsection{Ejercicio 7}
Los siguientes programas no compilan ¿por qué?\par
Intenta averiguar porqué no compila y posteriormente compila el código tal y como está para posteriormente obsevar las salidas de error del compilador.
\begin{verbatim}
\\Programa 1
public static void main(String[] args) {
}
public class Welcome {
    System.out.println("afternoon");
    System.out.println("morning");
}

\\Programa 2
public class Welcome {
   public void Main(String[] args) {
   System.out.println('Welcome to Java!);
   }
)
\\Programa 3
public class Hola{
    public static void main(){
    System.out.println('Welcome to Java!);
    }
}	
\\Programa 4
public class Hola{
   public static void main(String[] a){
   System.out.println('Welcome to Java!);
}
\end{verbatim}

\subsection{Ejercicio 8}
¿Cuales son los comandos para compilar y ejecutar un programa en Java, respectivamente?

\section{Javadoc}
\subsection{Ejercicio1}
Modifica tu programa \emph{HolaMundo} para que pueda genera documentación con javadoc.\\
Una vez realizado en \emph{Ubuntu}, copia el fichero de salida de compilación y comprueba su correcta ejecución en \emph{Windows}
\newpage

\subsection{Ejercicio2}
Dado el siguiente programa:
\begin{verbatim}
import java.util.ArrayList;
import java.util.Random;

/**
* Esta clase define objetos que contienen tantos enteros aleatorios entre 0 y 1000 como se le definen al crear un objeto
* @author: Mario R. Rancel
* @version: 22/09/2016/A
* @see <a href = "http://www.aprenderaprogramar.com" /> aprenderaprogramar.com – Didáctica en programación </a>
*/

public class SerieDeAleatoriosD {

    //Campos de la clase
    private ArrayList<Integer> serieAleatoria;
    /**
     * Constructor para la serie de números aleatorios
     * @param numeroItems El parámetro numeroItems define el número de elementos que va a tener la serie aleatoria
     */

    public SerieDeAleatoriosD (int numeroItems) {
        serieAleatoria = new ArrayList<Integer> ();
        for (int i=0; i<numeroItems; i++) {  serieAleatoria.add(0);  }
        System.out.println ("Serie inicializada. El número de elementos en la serie es: " + getNumeroItems() );
    } //Cierre del constructor

    /**
    * Método que devuelve el número de ítems (números aleatorios) existentes en la serie
    * @return El número de ítems (números aleatorios) de que consta la serie
    */

    public int getNumeroItems() { return serieAleatoria.size(); }

    /**
     * Método que genera la serie de números aleatorios
     */

    public void generarSerieDeAleatorios () {
        Random numAleatorio;
        numAleatorio = new Random ();
        for (int i=0; i < serieAleatoria.size(); i++) { serieAleatoria.set(i, numAleatorio.nextInt(1000) ); }
        System.out.print ("Serie generada! ");
    } //Cierre del método

} //Cierre de la clase y del ejemplo aprenderaprogramar.com
\end{verbatim}
\begin{itemize}
\item ¿Cúal es el nombre del fichero que contiene el ćodigo fuente?
\item Compila el programa y comprueba el fichero de salida
\item Intenta ejecutar el programa. ¿Qué ocurre?
\item Antes de generar la documentación, indica cuales comentarios no saldrán en la misma.
\item Genera la documentación.
\end{itemize}
\section{Codigo}
\subsection{Ejercicio 1}
Crea un programa que muestre por pantalla tu nombre cinco veces y la edad, en lineas diferentes.\\
Ejemplo:
\begin{verse}
Juan Gabriel Medina Río 22\\
Juan Gabriel Medina Río 22\\
Juan Gabriel Medina Río 22\\
Juan Gabriel Medina Río 22\\
Juan Gabriel Medina Río 22\\
\end{verse}
\subsection{Ejercicio 2}
Compila y ejecuta el siguiente código en Ubuntu:
\begin{verbatim}
public class Colores {
public static void main(String[] args) {
        String rojo = "\033[31m";
        String verde = "\033[32m";
        String naranja= "\033[33m";
        String azul = "\033[34m";
        String morado = "\033[35m";
        String blanco = "\033[37m";
        System.out.print(naranja + "mandarina" + verde + " hierba");
        System.out.print(naranja + " saltamontes" + rojo + " tomate");
        System.out.print(blanco + " sábanas" + azul + " cielo");
        System.out.print(morado + " nazareno" + azul + " mar");
        }
}
\end{verbatim}
\subsection{Ejercicio 3}
Escribe un programa que muestre tu horario de clase. Puedes usar espacios o tabuladores para alinear el texto. También puedes usar el mismo color para el mismo módulo.
\end{document}


