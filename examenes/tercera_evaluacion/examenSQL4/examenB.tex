\documentclass[4paper]{article}
\usepackage[spanish]{babel}
%\usepackage[ansinew]{inputenc}
\usepackage[utf8x]{inputenc}
%\usepackage[utf-8]{inputenc}
%\usepackage[T1]{fontenc}
\usepackage{graphicx}
\usepackage{multicol}
%\usepackage{longtable}
%\usepackage{array}
%\usepackage{multirow}
%\usepackage[latin1]{inputenc}
%\inputencoding{latin1}
%\usepackage{eurosym}

\renewcommand{\tablename}{Tabla}
\renewcommand{\S}{Base de datos y Java}
\author{Examen tercera evaluación}
\author{Tipo B}
\title{\textbf{\S}}
\date{\today}

\begin{document}
\maketitle 
%\tableofcontents
%\newpage
\section*{Creacion de la BD}
El ejercicio consiste en la creación de una base de datos llamada ExamenB, en un SGBD \emph{sqlite} que abarque la siguientes descripción:\par 
Queremos organizar los productos de una empresa de informática, para esto debemos controlar los siguientes aspectos:
\begin{itemize}
\item Un producto informático se caracteriza por su nombre, el precio y la categoría a la que pertenece, por ejemplo una tarjeta de sonido de 25 euros y pertenece a la categoría hardware, drivers tarjeta sonido 2 euros de la categoría software, sistema operativo windows 8, 250 euros de sistemas operativos, \dots
\item Un producto solo puede pertenecer a una categoría y en una categoría puede haber mas de un producto.
\item Las categorías se caracterizan únicamente por su nombre. 
\end{itemize}
Debes aportar el correpondiente diagrama ER, realizar un script para la creación de las correspondientes tablas, que debe contemplar la integridad referencial del \emph{SGBD} e introducir el nombre de diez productos informáticos y cuatro categorías. La asignación que hagas de productos a categorías es libre.\par 
Añade al script las siguientes características:
\begin{itemize}
\item Una \emph{vista} que nos de el nombre del producto, el precio  y el nombre de la categoría a la que pertenence.
\item Un \emph{trigger} que controle el historial de antiguos precios de los productos, es decir que cuando un producto modifique su precio, quede registrado el antiguo precio de la fecha de modificación.
\end{itemize}

\section*{Creacion de la aplicación}
Debemos crear una aplicación en \emph{Java} que nos permita hacer operaciones sobre la \emph{base de datos} anterior, las operaciones que debes realizar son:
\begin{itemize}
\item Insertar categorias, dado su nombre.
\item Cambiar el nombre de una categoría.
\item Borrar una categoría.
\item Aprovechando la vista anterior, una consulta que nos de una lista de productos  que incluya la categoría a la que pertenece.
\end{itemize}
Te aconsejo que crees dos clases \emph{Categoria} y \emph{Producto}, con los atributos que mapeen los atributos de ambas tablas en la \emph{BD}, con sus correpondeintes \emph{getters y setter}, constructor, \dots \par 
La \emph{BD} la debes incorporar a la carpeta raíz de tu workspace, de manera que la llamada a la BD la hagas solo con el nombre de dicha \emph{BD} sin indicar nada de ruta o path.
\par 
Comenta el código siempre que te sea posible, no se trata que uses \emph{javadoc}

\section*{Para los que trabajen en Windows}
No usar variables ni comentarios con caracteres que no sean \emph{ASCII}, es decir que sobre todo no utilices acentos ni eñes.

\section*{Criterios de evaluacion}
Los criterios de evaluación son los indicados a continuación:\par 
\vspace*{0.5cm}
\begin{tabular}{|c|c|}
\hline
\textbf{CRITERIO EVALUACION} & \textbf{PUNTUACION} \\
\hline
Creacion de tablas en el script & 1.5 pto.\\
\hline
Integridad referencial en la BD & 0.5 ptos.\\
\hline
Insercción datos & 0.5 ptos.\\
\hline
Creacción de la vista & 0.5 ptos.\\
\hline
Creacción del trigger & 0.5 ptos.\\
\hline
Uso del patrón Singleton para la conexión  & 2 ptos\\
\hline
No uso del patrón en la conexión & 1 ptos.\\
\hline
Uso DAO y DTO aplicación & 3 pto.\\
\hline 
No uso del patrón DAO y DTO pero si POO  & 2 ptos.\\
\hline
No uso del patrón DAO y DTO y sin POO & 1 pto.\\
\hline
Diagrama ER correcto & 0.5 ptos.\\
\hline
Diagrama UML  de las clases& 1 pto.\\
\hline
\end{tabular}
\section*{Subida de ficheros}
Debes incluir tres ficheros:
\begin{itemize}
\item El \emph{workspace} comprimido, en el que esté incluido la \emph{BD}, que quiere decir esto: que incorpores en la carpeta del worspace la \emph{BD} creada en \emph{sqlite}. También debe aparecer el diagrama UML en formato \emph{png o jpeg}.
\item El script de creación de la \emph{BD}.
\item El diagrama ER realizado en \emph{dia} o cualquier otro programa y lo exportas a un formato imagen.
\end{itemize}
\end{document}
