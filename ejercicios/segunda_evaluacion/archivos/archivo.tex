\documentclass[4paper]{article}
\usepackage[spanish]{babel}
%\usepackage[ansinew]{inputenc}
\usepackage[utf8x]{inputenc}
%\usepackage[utf-8]{inputenc}
%\usepackage[T1]{fontenc}
\usepackage{graphicx}
\usepackage{multicol}
%\usepackage{longtable}
%\usepackage{array}
%\usepackage{multirow}
%\usepackage[latin1]{inputenc}
%\inputencoding{latin1}

\renewcommand{\tablename}{Tabla}
\author{Manuel Molino Milla \and Luis Molina Garzón}
\title{\textbf{Programación}
\\I/O}
\date{\today}

\begin{document}

\maketitle

\subsection*{Ejercicio 1}
Escribe un programa que genere 20 números enteros comprendidos entre el 20 al 40 de forma aleatoria. Posteriormente el programa debe crear los siguientes fichero de acuerdo a las especificaciones siguientes:
\begin{description}
\item[numeros.dat] usando \emph{DataOutputStream}, \emph{BufferedOutputStream} y \emph{FileOutputStream}, este fichero es binario.
\item[numeros.txt] usando \emph{BufferedWriter}, este fichero es de texto por lo que debes tratar los números como cadenas.
\end{description}

\subsection*{Ejercicio 2}
Ahora debes leer los dos ficheros anteriores y mostrar por pantalla el valor medio de los datos guardados. \\
Usa para el primer caso \emph{DataInputStream} y para el segundo un \emph{BufferedReader}.

\subsection*{Ejercicio 3}
Repite el ejercicio usando un RandomAccesFile, debes crear dos programas uno que genere el fichero de salida y otro que lea dicho fichero y nos muestre en consola el valor medio de los datos.

\subsection*{Ejercicio 4}
Usando alguna de las clases de \emph{Reader} lee el archivo \emph{libros.csv} que contiene una lista de datos sobre libros entre los que incluye el título, nombre del autor e ISBN. Con esos datos realiza el siguiente ejercicio. 
\begin{itemize}
\item Una clase denominada \emph{Libro} con los atributos: titulo, autor e isbn. Añade los métodoos que necesites posteriormente.
\item En la misma clase que leas el archivo de entrada, crea una colección de objetos \emph{Libro} y añade cada objeto creado a la misma.
\item Escoge aleatoriamente diez objetos \emph{Libro} de esta colección y crea una nueva colección que los guarde, hay que tener en cuenta que los objeto \emph{Libro} no se pueden repetir en la nueva colección.
\item Guarda esta última lista en un fichero de datos denominado \emph{libros.dat}.
\item Posteriormente crea una clase que lea este fichero y muestre el contenido del mismo. Es decir los diez libros escogidos aleatoriamente.
\end{itemize}

\subsection*{Ejercicio 5}
Usando la clase  \emph{Scanner}  lee el archivo \emph{geografia.csv} que contiene una lista de datos sobre posiciones de ciertas localidades y realiza:
\begin{itemize}
\item Una clase denominada \emph{Localidad} con los atributos que aparecen en el archivo csv.
\item Crea una clase denominada \emph{Mapa} que tenga como atributo una colección de objetos \emph{Localidad}. No uses constructor y crea los siguientes métodos:
\begin{itemize}
\item getters y setters.
\item Un método que añada localidades a la colección.
\item Sobreescribe el método \emph{toString}
\item Un método que dado el nombre de la localidad nos devuelva un array con la longitud y latitud de la misma.
\end{itemize}
\item Crea una clase que lea el archivo de entrada, crea un objeto de tipo Mapa y añada la informacion obtenida del fichero.
\item Guarda dicho objeto en un fichero de datos.
\item Crea una clase que lea el fichero anterior y que mediante un Scanner pasa una localidad y obtén la longitud y latitud de la misma.
\end{itemize}

\subsection*{Ejercicio 6}
Realiza un programa que divida un fichero en distintos trozos mas pequeños. La sintáxis de ejecución sería:
\begin{quote}
\emph{java Leer ficheroEntrada numeroTrozos}
\end{quote}
Donde \emph{Leer} es es nombre del programa ejecutable en java, \emph{ficheroEntrada} es el fichero a dividir en trozos y \emph{numeroTrozos} es un numero que indica en el numero de partes en que se va a dividir el fichero.\\
Utiliza un fichero de texto como fichero de entrada.


\subsection*{Ejercicio 7}
Realiza el proceso contrario, es decir coge todos los trozos del fichero que has desmenuzado y vuelve a reensamblarlo en un nuevo fichero usando clases de I/O de java.

\end{document}
