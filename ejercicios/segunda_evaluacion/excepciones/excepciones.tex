\documentclass[4paper]{article}
\usepackage[spanish]{babel}
%\usepackage[ansinew]{inputenc}
\usepackage[utf8x]{inputenc}
%\usepackage[utf-8]{inputenc}
%\usepackage[T1]{fontenc}
\usepackage{graphicx}
\usepackage{multicol}
%\usepackage{longtable}
%\usepackage{array}
%\usepackage{multirow}
%\usepackage[latin1]{inputenc}
%\inputencoding{latin1}

\renewcommand{\tablename}{Tabla}
\author{Manuel Molino Milla \and Luis Molina Garzón}
\title{\textbf{Programación}
\\EXCEPCIONES}
\date{\today}

\begin{document}
\maketitle
%\tableofcontents
%\setlongtable 

\begin{abstract}
Intenta realizar los ejercicios aproximandote al máximo al paradigma de \emph{POO}. De todas formas es opcional, pero si muy recomendable su realización.
\end{abstract}


\subsection*{Ejercicio 1}
Crea una clase denominada Triangulo, que tenga como constructor un método con tres parámetros, que son la longitud de los lados del mismo. Crea una excepción denominada \emph{IlegalTrianguloExcepcion} que se lance cuando crear un triangulo no válido.
Los tres lados a, b y c de un triángulo deben satisfacer la desigualdad triangular: cada uno de los lados no puede ser más largo que la suma de los otros dos.
\subsection*{Ejercicio 2}
Crea dos clases denominada ConversionHexadecimal y ConversionBinaria que implementen los siguientes métodos:
\begin{description}
\item[hexADecimal(String hexString)] que dado un String que representa un número en hexadecimal, lo convierta a decimal.
\item[binarioAoDecimal (String binaryString)] que dado un String que representa un número binario, lo convierta a decimal
\end{description}
Añade las excepciones correspondientes para que no se permita introducir valores no válidos.\\
Crea dos programas denominados TestConversionBinario y TestConversionHexadecimal que compruebe el correcto funcionamiento de dicha clase.
\\
Para realizar el ejercicio no uses ninguna API que realice la conversión directa. Implementa el algoritmo.


\end{document}
