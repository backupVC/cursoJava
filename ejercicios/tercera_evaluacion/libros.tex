\documentclass[a4paper,spanish]{article}
\usepackage[spanish]{babel}
\usepackage[ansinew]{inputenc}
\usepackage[T1]{fontenc}
\usepackage{graphicx}
\usepackage{multicol}
\usepackage{longtable}
\usepackage{array}
\usepackage{multirow}

\renewcommand{\tablename}{Tabla}
\renewcommand{\S}{Sqlite }
\author{Manuel Molino Milla \and Luis Molina Garz�n}
\title{\textbf{\S}}
\date{\today}

\begin{document}
\maketitle

De acuerdo al ejercicio final propuesto en el tema anterior (\emph{sqlite}) que era el siguiente:

El ejercicio final consistia en la realizacion de una BD para inventariar libros y permitir su posterior prestamo. Para esto hay que tener en cuenta los siguientes datos:
\begin{itemize}
\item Los libros pueden pertenecer a una unica categoria, que puede ser: Bases de datos, programacion, redes, ofimatica, hardware, seguridad, aplicaciones web,  sistemas operativos o miscelanea.
\item De los libros debemos conocer el nombre, autor y editorial.
\item De los usuarios su nombre y apellidos.
\item Los libros pueden ser prestados, lo unico que tenemos que conocer es la persona que lo retira y la fecha de prestamo. 
\item Prepara las siguientes consultas, si es necesario prepara vistas para las mismas.
\begin{enumerate}
\item El numero de libros prestados.
\item El listado de libros prestados.
\item El listado de libros por categorias.
\item El listado de libros prestados por categorias.
\item El listado de libros no prestados por categorias.
\item El listado de libros no prestados.
\item Los libros prestados a un usuario.
\item Listado de usuarios.
\end{enumerate}
\item Prepara dos triggers para dar de baja a usuarios y libros.
\end{itemize}

La identificacion de las categorias se realiza de la siguiente manera:\par
\vspace*{0.3cm}
\begin{center}
\begin{tabular}{|c|c|}
\hline
ID CATEGORIA & CATEGORIA\\
\hline
1	&	Base de datos\\
\hline
2	&	Programacion\\
\hline
3	&	Redes\\
\hline
4	&	Ofimatica\\
\hline
5	&	Hardware\\
\hline
6	&	Seguridad\\
\hline
7	&	Aplicaciones Web\\
\hline
8	&	Sistemas Operativos\\
\hline 
9	&	Miscelanea\\
\hline
\end{tabular}
\vspace*{0.5cm}
\end{center}
Realiza un programa en Java que resuelva las siguientes especificaciones:
\begin{itemize}
\item Debes crear una clase que cargue los datos de las tablas libros del fichero \emph{libros.sql}
\item Crear una clase que cargue en la BD las categorias con su correspondiente \emph{id}
\item Crear una clase que cargue 10 usuarios que tu te inventes.
\item Puedes obviar las clases anteriores e introducir los datos directamente en la BD.
\item Crear una clase que resuelva la consulta de todos los datos de los libros que nos de el titulo, nombre y autor.
\item Crear una consulta que nos de los datos de los usuarios (nombres y apellidos).
\item Crear una clase que resuelva la consulta de los libros que pertenezcan a una categoria dada.
\item Crear una clase que gestione el prestamo de libros.
\item Otra clase que gestione la devolucion de libros.
\item Crea una clase que nos de los libros prestados actualmente.
\item Otra clase que nos de un historial de libros prestados.
\item Por ultimo otra clase que facilite la  baja a libros y su posterior consulta.

\end{itemize}


\end{document}
